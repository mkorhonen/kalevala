% Neljäs runo

\chapter*{Neljäs: Runo}

\colorini{A}ino, neito nuori, sisar nuoren Joukahaisen,        \\
läksi luutoa lehosta, vastaksia varvikosta.                    \\
Taittoi vastan taatollensa, toisen taittoi maammollensa,       \\
kokoeli kolmannenki verevälle veijollensa.                     \\
                                                               \\
Jo astui kohin kotia, lepikköä leuhautti.                      \\
Tuli vanha Väinämöinen; näki neitosen lehossa,                 \\
hienohelman heinikössä. Sanan virkkoi, noin nimesi:            \\
"Eläpä muille, neiti nuori, kuin minulle, neiti nuori,         \\
kanna kaulanhelmilöitä, rinnanristiä rakenna,                  \\
pane päätä palmikolle, sio silkillä hivusta!"                  \\
                                                               \\
Neiti tuon sanoiksi virkki: "En sinulle enkä muille            \\
kanna rinnanristilöitä, päätä silkillä sitaise.                \\
Huoli on haahen haljakoista, vehnän viploista valita;          \\
asun kaioissa sovissa, kasvan leivän kannikoissa               \\
tykönä hyvän isoni, kanssa armahan emoni."                     \\
                                                               \\
Riisti ristin rinnaltansa, sormukset on sormestansa,           \\
helmet kaulasta karisti, punalangat päänsä päältä,             \\
jätti maalle maan hyviksi, lehtohon lehon hyviksi.             \\
Meni itkien kotihin, kallotellen kartanolle.                   \\
                                                               \\
Iso istui ikkunalla, kirvesvartta kirjoavi:                    \\
"Mitä itket, tyttö raukka, tyttö raukka, neito nuori?"         \\
                                                               \\
"Onpa syytä itkeäni, vaivoja valittoani!                       \\
Sitä itken, taattoseni, sitä itken ja valitan:                 \\
kirpoi risti rinnaltani, kaune vyöstäni karisi,                \\
rinnalta hopearisti, vaskilangat vyöni päästä."                \\
                                                               \\
Veljensä veräjän suulla vemmelpuuta veistelevi:                \\
"Mitä itket, sisko raukka, sisko raukka, neito nuori?"         \\
                                                               \\
"Onpa syytä itkeäni, vaivoja valittoani!                       \\
Sitä itken, veikko rukka, sitä itken ja valitan:               \\
kirpoi sormus sormestani, helmet kaulasta katosi,              \\
kullansormus sormestani, kaulasta hopeahelmet."                \\
                                                               \\
Sisko sillan korvasella vyötä kullaista kutovi:                \\
"Mitä itket, sisko raukka, sisko raukka, neito nuori?"         \\
                                                               \\
"Onpa syytä itkijällä, vaivoja vetistäjällä!                   \\
Sitä itken, sisko rukka, sitä itken ja valitan:                \\
kirpoi kullan kulmiltani, hopeat hivuksiltani,                 \\
sinisilkit silmiltäni, punanauhat pääni päältä."               \\
                                                               \\
Emo aitan portahalla kuoretta kokoelevi:                       \\
"Mitä itket, tytti raukka, tyttö raukka, neito nuori?"         \\
                                                               \\
"Oi on maammo, kantajani, oi emo, imettäjäni!                  \\
Onp' on syitä synke'itä, apeita ani pahoja!                    \\
Sitä itken, äiti rukka, sitä, maammoni, valitan:               \\
läksin luutoa lehosta, vastanpäitä varvikosta.                 \\
Taitoin vastan taatolleni, toisen taitoin maammolleni,         \\
kokoelin kolmannenki verevälle veijolleni.                     \\
Aloin astua kotihin; astuinpa läpi ahosta:                     \\
Osmoinen orosta virkkoi, Kalevainen kaskesmaalta:              \\
'Eläpä muille, neiti rukka, kuin minulle, neiti rukka,         \\
kanna kaulanhelmilöitä, rinnanristiä rakenna,                  \\
pane päätä palmikolle, sio silkillä hivusta!'                  \\
                                                               \\
"Riistin ristin rinnaltani, helmet kaulasta karistin,          \\
sinilangat silmiltäni, punalangat pääni päältä,                \\
heitin maalle maan hyviksi, lehtohon lehon hyviksi.            \\
Itse tuon sanoiksi virkin: 'En sinulle enkä muille             \\
kanna rinnanristiäni, päätä silkillä sitaise.                  \\
Huoli en haahen haljakoista, vehnän viploista valita;          \\
asun kaioissa sovissa, kasvan leivän kannikoissa               \\
tykönä hyvän isoni, kanssa armahan emoni.'"                    \\
                                                               \\
Emo tuon sanoiksi virkki, lausui vanhin lapsellensa:           \\
"Elä itke, tyttäreni, nuorna saamani, nureksi!                 \\
Syö vuosi suloa voita: tulet muita vuolahampi;                 \\
toinen syö sianlihoa: tulet muita sirkeämpi;                   \\
kolmas kuorekokkaroita: tulet muita kaunihimpi.                \\
Astu aittahan mäelle  aukaise parahin aitta!                   \\
                                                               \\
Siell' on arkku arkun päällä, lipas lippahan lomassa.          \\
Aukaise parahin arkku, kansi kirjo kimmahuta:                  \\
siin' on kuusi kultavyötä, seitsemän sinihamoista.             \\
Ne on Kuuttaren kutomat, Päivättären päättelemät.              \\
                                                               \\
"Ennen neinnä ollessani, impenä eläessäni                      \\
läksin marjahan metsälle, alle vaaran vaapukkahan.             \\
Kuulin Kuuttaren kutovan, Päivättären kehreävän                \\
sinisen salon sivulla, lehon lemmen liepehellä.                \\
                                                               \\
"Minä luoksi luontelime, likelle lähentelime.                  \\
Aloinpa anella noita, itse virkin ja sanelin:                  \\
'Anna, Kuutar, kultiasi, Päivätär, hope'itasi                  \\
tälle tyhjälle tytölle, lapsellen anelijalle!'                 \\
                                                               \\
"Antoi Kuutar kultiansa, Päivätär hope'itansa.                 \\
Minä kullat kulmilleni, päälleni hyvät hopeat!                 \\
Tulin kukkana kotihin, ilona ison pihoille.                    \\
                                                               \\
"Kannoin päivän, kannoin toisen. Jo päivänä kolmantena         \\
riisuin kullat kulmiltani, päältäni hyvät hopeat,              \\
vein ne aittahan mäelle, panin arkun kannen alle:              \\
siit' on asti siellä ollut ajan kaiken katsomatta.             \\
                                                               \\
"Sio nyt silkit silmillesi, kullat kulmille kohota,            \\
kaulahan heleät helmet, kullanristit rinnoillesi!              \\
Pane paita palttinainen, liitä liinan-aivinainen,              \\
Hame verkainen vetäise, senp' on päälle silkkivyöhyt,          \\
sukat sulkkuiset koreat, kautokengät kaunokaiset!              \\
Pääsi kääri palmikolle, silkkinauhoilla sitaise,               \\
sormet kullansormuksihin, käet kullankäärylöihin!              \\
                                                               \\
"Niin tulet tupahan tuolta, astut aitasta sisälle              \\
sukukuntasi suloksi, koko heimon hempeäksi:                    \\
kulet kukkana kujilla, vaapukkaisena vaellat,                  \\
ehompana entistäsi, parempana muinaistasi."                    \\
                                                               \\
Sen emo sanoiksi virkki, senp' on lausui lapsellensa.          \\
Ei tytär totellut tuota, ei kuullut emon sanoja;               \\
meni itkien pihalle, kaihoellen kartanolle.                    \\
Sanovi sanalla tuolla, lausui tuolla lausehella:               \\
                                                               \\
"Miten on mieli miekkoisien, autuaallisten ajatus?             \\
Niinp' on mieli miekkoisien, autuaallisten ajatus,             \\
kuin on vellova vetonen eli aalto altahassa.                   \\
Mitenpä poloisten mieli, kuten allien ajatus?                  \\
Niinpä on poloisten mieli, niinpä allien ajatus,               \\
kuin on hanki harjun alla, vesi kaivossa syvässä.              \\
                                                               \\
"Usein nyt minun utuisen, use'in, utuisen lapsen,              \\
mieli kulkevi kulossa, vesakoissa viehkuroivi,                 \\
nurmessa nuhaelevi, pensahassa piehtaroivi;                    \\
mieli ei tervoa parempi, syän ei syttä valkeampi.              \\
                                                               \\
"Parempi minun olisi, parempi olisi ollut                      \\
syntymättä, kasvamatta, suureksi sukeumatta                    \\
näille päiville pahoille, ilmoille ilottomille.                \\
Oisin kuollut kuusiöisnä, kaonnut kaheksanöisnä,               \\
oisi en paljoa pitänyt: vaaksan palttinapaloa,                 \\
pikkaraisen pientaretta, emon itkua vähäisen,                  \\
ison vieläki vähemmän, veikon ei väheäkänä."                   \\
                                                               \\
Itki päivän, itki toisen. Sai emo kyselemähän:                 \\
"Mitä itket, impi rukka, kuta, vaivainen, valitat?"            \\
                                                               \\
"Sitä itken, impi rukka, kaiken aikani valitan,                \\
kun annoit minun poloisen, oman lapsesi lupasit,               \\
käskit vanhalle varaksi, ikäpuolelle iloksi,                   \\
turvaksi tutisevalle, suojaksi sopenkululle.                   \\
Oisit ennen käskenynnä alle aaltojen syvien                    \\
sisareksi siikasille, veikoksi ve'en kaloille!                 \\
Parempi meressä olla, alla aaltojen asua                       \\
sisarena siikasilla, veikkona ve'en kaloilla,                  \\
kuin on vanhalla varana, turvana tutisijalla,                  \\
sukkahansa suistujalla, karahkahan kaatujalla."                \\
                                                               \\
Siitä astui aittamäelle, astui aittahan sisälle.               \\
Aukaisi parahan arkun, kannen kirjo kimmahutti:                \\
löysi kuusi kultavyötä, seitsemän sinihametta;                 \\
ne on päällensä pukevi, varrellensa valmistavi.                \\
Pani kullat kulmillensa, hopeat hivuksillensa,                 \\
sinisilkit silmillensä, punalangat päänsä päälle.              \\
                                                               \\
Läksi siitä astumahan ahon poikki, toisen pitkin;              \\
vieri soita, vieri maita, vieri synkkiä saloja.                \\
Itse lauloi mennessänsä, virkkoi vieriellessänsä:              \\
"Syäntäni tuimelevi, päätäni kivistelevi.                      \\
Eikä tuima tuimemmasti, kipeämmästi kivistä,                   \\
jotta, koito, kuolisinki, katkeaisinki, katala,                \\
näiltä suurilta suruilta, ape'ilta miel'aloilta.               \\
                                                               \\
"Jo oisi minulla aika näiltä ilmoilta eritä,                   \\
aikani Manalle mennä, ikä tulla Tuonelahan:                    \\
ei minua isoni itke, ei emo pane pahaksi,                      \\
ei kastu sisaren kasvot, veikon silmät vettä vuoa,             \\
vaikka vierisin vetehen, kaatuisin kalamerehen                 \\
alle aaltojen syvien, päälle mustien murien."                  \\
                                                               \\
Astui päivän, astui toisen. Päivänäpä kolmantena               \\
ennätti meri etehen, ruokoranta vastahansa:                    \\
tuohon yöhyt yllättävi, pimeä piättelevi.                      \\
                                                               \\
Siinä itki impi illan, kaikerteli kaiken yötä                  \\
rannalla vesikivellä, laajalla lahen perällä.                  \\
Aamulla ani varahin katsoi tuonne niemen päähän:               \\
kolme oli neittä niemen päässä ... ne on merta                 \\
kylpemässä!                                                    \\
Aino neiti neljänneksi, vitsan varpa viienneksi!               \\
                                                               \\
Heitti paitansa pajulle, hamehensa haapaselle,                 \\
sukkansa sulalle maalle, kenkänsä vesikivelle,                 \\
helmet hietarantaselle, sormukset somerikolle.                 \\
                                                               \\
Kivi oli kirjava selällä, paasi kullan paistavainen:           \\
kiistasi kivellen uia, tahtoi paaelle paeta.                   \\
                                                               \\
Sitte sinne saatuansa asetaiksen istumahan                     \\
kirjavaiselle kivelle, paistavalle paaterelle:                 \\
kilahti kivi vetehen, paasi pohjahan pakeni,                   \\
neitonen kiven keralla, Aino paaen palleassa.                  \\
                                                               \\
Siihenpä kana katosi, siihen kuoli impi rukka.                 \\
Sanoi kerran kuollessansa, virkki vielä vierressänsä:          \\
"Menin merta kylpemähän, sainp' on uimahan selälle;            \\
sinne mä, kana, katosin, lintu, kuolin liian surman:           \\
elköhön minun isoni sinä ilmoisna ikänä                        \\
vetäkö ve'en kaloja tältä suurelta selältä!                    \\
                                                               \\
"Läksin rannalle pesohon, menin merta kylpemähän;              \\
sinne mä, kana, katosin, lintu, kuolin liian surman:           \\
elköhön minun emoni sinä ilmoisna ikänä                        \\
panko vettä taikinahan laajalta kotilahelta!                   \\
                                                               \\
"Läksin rannalle pesohon, menin merta kylpemähän;              \\
sinne mä, kana, katosin, lintu, kuolin liian surman:           \\
elköhönp' on veikkoseni sinä ilmoisna ikänä                    \\
juottako sotaoritta rannalta merelliseltä!                     \\
                                                               \\
"Läksin rannalle pesohon, menin merta kylpemähän;              \\
sinne mä, kana, katosin, lintu, kuolin liian surman:           \\
elköhönp' on siskoseni sinä ilmoisna ikänä                     \\
peskö tästä silmiänsä kotilahen laiturilta!                    \\
Mikäli meren vesiä, sikäli minun veriä;                        \\
mikäli meren kaloja, sikäli minun lihoja;                      \\
mikä rannalla risuja, se on kurjan kylkiluita;                 \\
mikä rannan heinäsiä, se hivusta hierottua."                   \\
                                                               \\
Se oli surma nuoren neien, loppu kaunihin kanasen...           \\
                                                               \\
Kukas nyt sanan saatantahan, kielikerran kerrontahan           \\
neien kuuluhun kotihin, kaunihisen kartanohon?                 \\
                                                               \\
Karhu sanan saatantahan, kielikerran kerrontahan!              \\
Ei karhu sanoa saata: lehmikarjahan katosi.                    \\
                                                               \\
Kukas sanan saatantahan, kielikerran kerrontahan               \\
neien kuuluhun kotihin, kaunihisen kartanohon?                 \\
                                                               \\
Susi sanan saatantahan, kielikerran kerrontahan!               \\
Ei susi sanoa saata: lammaskarjahan katosi.                    \\
                                                               \\
Kukas sanan saatantahan, kielikerran kerrontahan               \\
neien kuuluhun kotihin, kaunihisen kartanohon?                 \\
                                                               \\
Repo sanan saatantahan, kielikerran kerrontahan!               \\
Ei repo sanoa saata: hanhikarjahan katosi.                     \\
                                                               \\
Kukas sanan saatantahan, kielikerran kerrontahan               \\
neien kuuluhun kotihin, kaunihisen kartanohon?                 \\
                                                               \\
Jänö sanan saatantahan, kielikerran kerrontahan!               \\
Jänis varman vastaeli: "Sana ei miehe'en katoa!"               \\
                                                               \\
Läksi jänis juoksemahan, pitkäkorva piippomahan,               \\
vääräsääri vääntämähän, ristisuu ripottamahan                  \\
neien kuuluhun kotihin, kaunihisen kartanohon.                 \\
                                                               \\
Juoksi saunan kynnykselle; kyykistäikse kynnykselle:           \\
sauna täynnä neitosia, vasta käessä vastoavat:                 \\
"Saitko, kiero, keittimiksi, paltsasilmä, paistimiksi,         \\
isännällen iltaseksi, emännällen eineheksi,                    \\
tyttären välipaloiksi, pojan puolipäiväseksi?"                 \\
                                                               \\
Jänis saattavi sanoa, kehräsilmä kerskaella:                   \\
"Liepä lempo lähtenynnä kattiloihin kiehumahan!                \\
Läksin sanan saatantahan, kielikerran kerrontahan:             \\
jop' on kaunis kaatununna, tinarinta riutununna,               \\
sortununna hopeasolki, vyö vaski valahtanunna:                 \\
mennyt lietohon merehen, alle aavojen syvien,                  \\
sisareksi siikasille, veikoksi ve'en kaloille."                \\
                                                               \\
Emo tuosta itkemähän, kyynelvierus vieremähän.                 \\
Sai siitä sanelemahan, vaivainen valittamahan:                 \\
"Elkätte, emot poloiset, sinä ilmoisna ikänä                   \\
tuuitelko tyttäriä, lapsianne liekutelko                       \\
vastoin mieltä miehelähän, niinkuin mie, emo poloinen,         \\
tuuittelin tyttöjäni, kasvatin kanasiani!"                     \\
                                                               \\
Emo itki, kyynel vieri: vieri vetrehet vetensä                 \\
sinisistä silmistänsä poloisille poskillensa.                  \\
                                                               \\
Vieri kyynel, vieri toinen: vieri vetrehet vetensä             \\
poloisilta poskipäiltä ripe'ille rinnoillensa.                 \\
                                                               \\
Vieri kyynel, vieri toinen: vieri vetrehet vetensä             \\
ripe'iltä rinnoiltansa hienoisille helmoillensa.               \\
                                                               \\
Vieri kyynel, vieri toinen: vieri vetrehet vetensä             \\
hienoisilta helmoiltansa punasuille sukkasille.                \\
                                                               \\
Vieri kyynel, vieri toinen: vieri vetrehet vetensä             \\
punasuilta sukkasilta kultakengän kautosille.                  \\
                                                               \\
Vieri kyynel, vieri toinen: vieri vetrehet vetensä             \\
kultakengän kautosilta maahan alle jalkojensa;                 \\
vieri maahan maan hyväksi, vetehen ve'en hyväksi.              \\
                                                               \\
Ve'et maahan tultuansa alkoivat jokena juosta:                 \\
kasvoipa jokea kolme itkemistänsä vesistä,                     \\
läpi päänsä lähtemistä, alta kulman kulkemista.                \\
                                                               \\
Kasvoipa joka jokehen kolme koskea tulista,                    \\
joka kosken kuohumalle kolme luotoa kohosi,                    \\
joka luo'on partahalle kunnas kultainen yleni;                 \\
kunki kunnahan kukulle kasvoi kolme koivahaista,               \\
kunki koivun latvasehen kolme kullaista käkeä.                 \\
                                                               \\
Sai käköset kukkumahan. Yksi kukkui: "lemmen,                  \\
lemmen!"                                                       \\
Toinen kukkui: "sulhon, sulhon!" Kolmas kukkui:                \\
"auvon, auvon!"                                                \\
                                                               \\
Kuka kukkui: "lemmen, lemmen!" sep' on kukkui kuuta            \\
kolme                                                          \\
lemmettömälle tytölle, meressä makoavalle.                     \\
                                                               \\
Kuka kukkui: "sulhon, sulhon!" sep' on kukkui kuusi            \\
kuuta                                                          \\
sulholle sulottomalle, ikävissä istuvalle.                     \\
                                                               \\
Kuka kukkui: "auvon, auvon!" se kukkui ikänsä kaiken           \\
auvottomalle emolle, iän päivät itkevälle.                     \\
                                                               \\
Niin emo sanoiksi virkki kuunnellessansa käkeä:                \\
"Elköhön emo poloinen kauan kuunnelko käkeä!                   \\
Kun käki kukahtelevi, niin syän sykähtelevi,                   \\
itku silmähän tulevi, ve'et poskille valuvi,                   \\
hereämmät herne-aarta, paksummat pavun jyveä:                  \\
kyynärän ikä kuluvi, vaaksan varsi vanhenevi,                  \\
koko ruumis runnahtavi kuultua kevätkäkösen."                  \\