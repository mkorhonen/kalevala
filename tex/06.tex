% Kuudes runo

\chapter*{Kuudes: Runo}

\colorini{V}{aka vanha Väinämöinen} lähteäksensä käkesi       \\
tuonne kylmähän kylähän, pimeähän Pohjolahan.               \\
                                                            \\
Otti olkisen orihin, hernevartisen hevosen,                 \\
pisti suitset kullan suuhun, päitsensä hopean päähän:       \\
itse istuvi selälle, löihe reisin ratsahille.               \\
Ajoa hyryttelevi, matkoansa mittelevi                       \\
orihilla olkisella, hernevarrella hevolla.                  \\
                                                            \\
Ajoi Väinölän ahoja, Kalevalan kankahia:                    \\
hepo juoksi, matka joutui, koti jääpi, tie lyheni.          \\
Jo ajoi meren selälle, ulapalle aukealle                    \\
kapioisen kastumatta, vuohisen vajoumatta.                  \\
                                                            \\
Olipa nuori Joukahainen, laiha poika lappalainen.           \\
Piti viikoista vihoa, ylen kauaista kaetta                  \\
kera vanhan Väinämöisen, päälle laulajan ikuisen.           \\
                                                            \\
Laativi tulisen jousen, jalon kaaren kaunistavi:            \\
kaaren rauasta rakenti, vaskesta selän valavi;              \\
noita on kullalla kuvaili, hopealla huolitteli.             \\
                                                            \\
Mistä siihen nauhan saapi, kusta jäntehen tapasi?           \\
Hiien hirven suoniloista, Lemmon liinanuorasista!           \\
                                                            \\
Sai kaaren kanineheksi, jousen varsin valmihiksi.           \\
Kaari on kaunihin näköinen, jousi jonki maksavainen:        \\
hevonen selällä seisoi, varsa juoksi vartta myöten,         \\
kapo kaarella makasi, jänö jäntimen sijassa.                \\
                                                            \\
Vuoli piiliä pinosen, kolmisulkia kokosen:                  \\
varret tammesta vanuvi, päät tekevi tervaksesta.            \\
Minkä saapi valmihiksi, sen sitte sulittelevi               \\
pääskyn pienillä sulilla, varpusen vivustimilla.            \\
                                                            \\
Karkaeli nuoliansa, puretteli piiliänsä                     \\
maon mustissa mujuissa, käärmehen kähyverissä.              \\
                                                            \\
Sai vasamat valmihiksi, jousen jänniteltäväksi.             \\
Siitä vuotti Väinämöistä, saavaksi suvantolaista;           \\
vuotti illan, vuotti aamun, vuotti kerran keskipäivän.      \\
                                                            \\
Viikon vuotti Väinämöistä, viikon vuotti, ei väsynyt,       \\
istuellen ikkunoissa, valvoen vajojen päissä,               \\
kuunnellen kujan perällä, vahtaellen vainiolla,             \\
viini nuolia selässä, hyvä kaari kainalossa.                \\
                                                            \\
Vuotteli ulompanaki, talon toisen tuolla puolla:            \\
nenässä tulisen niemen, tulikaiskun kainalossa,             \\
korvalla tulisen kosken, pyhän virran viertimellä.          \\
                                                            \\
Niin päivänä muutamana, huomenna moniahana                  \\
loi silmänsä luotehelle, käänti päätä päivän alle;          \\
keksi mustasen merellä, sinerväisen lainehilla:             \\
"Onko se iässä pilvi, päivän koite koillisessa?"            \\
                                                            \\
Ei ollut iässä pilvi, päivän koite koillisessa:             \\
oli vanha Väinämöinen, laulaja iän-ikuinen,                 \\
matkoava Pohjolahan, kulkeva Pimentolahan                   \\
orihilla olkisella, hernevarrella hevolla.                  \\
                                                            \\
Tuop' on nuori Joukahainen, laiha poika lappalainen,        \\
jou'utti tulisen jousen, koppoi kaaren kaunihimman          \\
pään varalle Väinämöisen, surmaksi suvantolaisen.           \\
                                                            \\
Ennätti emo kysyä, vanhempansa tutkaella:                   \\
"Kellen jousta jouahutat, kaarta rauta rauahutat?"          \\
                                                            \\
Tuop' on nuori Joukahainen sanan virkkoi, noin nimesi:      \\
"Tuohon jousta jouahutan, kaarta rauta rauahutan:           \\
pään varalle Väinämöisen, surmaksi suvantolaisen.           \\
Ammun vanhan Väinämöisen, lasken laulajan ikuisen           \\
läpi syämen, maksan kautta, halki hartiolihojen."           \\
                                                            \\
Emo kielti ampumasta, emo kielti ja epäsi:                  \\
"Elä ammu Väinämöistä, kaota kalevalaista!                  \\
Väinö on sukua suurta: lankoni sisaren poika.               \\
                                                            \\
"Ampuisitko Väinämöisen, kaataisit kalevalaisen,            \\
ilo ilmalta katoisi, laulu maalta lankeaisi.                \\
Ilo on ilmalla parempi, laulu maalla laatuisampi,           \\
kuin onpi Manalan mailla, noilla Tuonelan tuvilla."         \\
                                                            \\
Tuossa nuori Joukahainen jo vähän ajattelevi,               \\
pikkuisen piättelevi: käsi käski ampumahan,                 \\
käsi käski, toinen kielti, sormet suoniset pakotti.         \\
                                                            \\
Virkki viimeinki sanoiksi, itse lausui, noin nimesi:        \\
"Kaotkohot jos kahesti kaikki ilmaiset ilomme,              \\
kaikki laulut langetkohot! Varsin ammun, en varanne."       \\
                                                            \\
Jännitti tulisen jousen, veti vaskisen vekaran              \\
vasten polvea vasenta, jalan alta oikeansa.                 \\
Veti viinestä vasaman, sulan kolmikoipisesta,               \\
otti nuolen orhe'imman, valitsi parahan varren;             \\
tuon on juonelle asetti, liitti liinajäntehelle.            \\
                                                            \\
Oikaisi tulisen jousen olallehen oikealle,                  \\
asetaiksen ampumahan, ampumahan Väinämöistä.                \\
Itse tuon sanoiksi virkki: "Iske nyt, koivuinen sakara,     \\
petäjäinen selkä, lyö'ös, jänne liina, lippaellos!          \\
Min käsi alentanehe, sen nuoli ylentäköhön;                 \\
min käsi ylentänehe, sen nuoli alentakohon!"                \\
                                                            \\
Lekahutti liipaisinta, ampui nuolen ensimäisen:             \\
se meni kovan ylätse, päältä pään on taivahalle,            \\
pilvihin pirajavihin, hattaroihin pyörivihin.               \\
                                                            \\
Toki ampui, ei totellut. Ampui toisen nuoliansa:            \\
se meni kovan alatse, alaisehen maa-emähän;                 \\
tahtoi maa manalle mennä, hietaharju halkiella.             \\
                                                            \\
Ampui kohta kolmannenki: kävi kohti kolmannesti,            \\
sapsohon sinisen hirven alta vanhan Väinämöisen;            \\
ampui olkisen orihin, hernevartisen hevosen                 \\
läpi länkiluun lihoista, kautta kainalon vasemman.          \\
                                                            \\
Siitä vanha Väinämöinen sormin suistuvi sulahan,            \\
käsin kääntyi lainehesen, kourin kuohu'un kohahti           \\
selästä sinisen hirven, hernevartisen hevosen.              \\
                                                            \\
Nousi siitä suuri tuuli, aalto ankara merellä;              \\
kantoi vanhan Väinämöisen, uitteli ulomma maasta            \\
noille väljille vesille, ulapoille auke'ille.               \\
                                                            \\
Siitä nuori Joukahainen itse kielin kerskaeli:              \\
"Et sinä, vanha Väinämöinen, enämpi elävin silmin           \\
sinä ilmoisna ikänä, kuuna kullan valkeana                  \\
astu Väinölän ahoja, Kalevalan kankahia!                    \\
                                                            \\
"Kupli nyt siellä kuusi vuotta, seuro seitsemän kesyttä,    \\
karehi kaheksan vuotta noilla väljillä vesillä,             \\
lake'illa lainehilla: vuotta kuusi kuusipuuna,              \\
seitsemän petäjäpuuna, kannon pölkkynä kaheksan!"           \\
                                                            \\
Siitä pistihe sisälle. Sai emo kysyneheksi:                 \\
"Joko ammuit Väinämöisen, kaotit Kalevan poian?"            \\
                                                            \\
Tuop' on nuori Joukahainen sanan vastahan sanovi:           \\
"Jo nyt ammuin Väinämöisen ja kaaoin kalevalaisen,          \\
loin on merta luutimahan, lainetta lakaisemahan.            \\
Tuohon lietohon merehen, aivan aaltojen sekahan             \\
sortui ukko sormillehen, kääntyi kämmenyisillehen;          \\
siitä kyykertyi kylelle, selällehen seisottihe              \\
meren aaltojen ajella, meren tyrskyn tyyräellä."            \\
                                                            \\
Tuon emo sanoiksi virkki: "Pahoin teit sinä poloinen,       \\
kun on ammuit Väinämöisen, kaotit kalevalaisen,             \\
Suvantolan suuren miehen, Kalevalan kaunihimman!"           \\