% Kolmas runo

\chapter*{Kolmas: Runo}

\colorini{V}{aka vanha Väinämöinen} elelevi aikojansa           \\
noilla Väinölän ahoilla, Kalevalan kankahilla.                \\
Laulelevi virsiänsä, laulelevi, taitelevi.                    \\
                                                              \\
Lauloi päivät pääksytysten, yhytysten yöt saneli              \\
muinaisia muisteloita, noita syntyjä syviä,                   \\
joit' ei laula kaikki lapset, ymmärrä yhet urohot             \\
tällä inhalla iällä, katovalla kannikalla.                    \\
                                                              \\
Kauas kuuluvi sanoma, ulos viestit vierähtävät                \\
Väinämöisen laulannasta, urohon osoannasta.                   \\
Viestit vierähti suvehen, sai sanomat Pohjolahan.             \\
                                                              \\
Olipa nuori Joukahainen, laiha poika lappalainen.             \\
Se kävi kylässä kerran; kuuli kummia sanoja,                  \\
lauluja laeltavaksi, parempia pantavaksi                      \\
noilla Väinölän ahoilla, Kalevalan kankahilla,                \\
kuin mitä itseki tiesi, oli oppinut isolta.                   \\
                                                              \\
Tuo tuosta kovin pahastui, kaiken aikansa kaehti              \\
Väinämöistä laulajaksi paremmaksi itseänsä.                   \\
Jo tuli emonsa luoksi, luoksi valtavanhempansa.               \\
Lähteäksensä käkesi, tullaksensa toivotteli                   \\
noille Väinölän tuville kera Väinön voitteloille.             \\
                                                              \\
Iso kielti poikoansa, iso kielti, emo epäsi                   \\
lähtemästä Väinölähän kera Väinön voitteloille:               \\
"Siellä silma lauletahan, lauletahan, lausitahan              \\
suin lumehen, päin vitihin, kourin ilmahan kovahan,           \\
käsin kääntymättömäksi, jaloin liikkumattomaksi."             \\
                                                              \\
Sanoi nuori Joukahainen: "Hyväpä isoni tieto,                 \\
emoni sitäi parempi, oma tietoni ylinnä.                      \\
Jos tahon tasalle panna, miesten verroille vetäitä,           \\
itse laulan laulajani, sanelen sanelijani:                    \\
laulan laulajan parahan pahimmaksi laulajaksi,                \\
jalkahan kiviset kengät, puksut puiset lantehille,            \\
kiviriipan rinnan päälle, kiviharkon hartioille,              \\
kivihintahat kätehen, päähän paatisen kypärän."               \\
                                                              \\
Siitä läksi, ei totellut. Otti ruunansa omansa,               \\
jonka turpa tulta iski, säkeniä säärivarret;                  \\
valjasti tulisen ruunan korjan kultaisen etehen.              \\
Itse istuvi rekehen, kohennaikse korjahansa,                  \\
iski virkkua vitsalla, heitti helmiruoskasella.               \\
Läksi virkku vieremähän, hevonen helettämähän.                \\
                                                              \\
Ajoa suhuttelevi. Ajoi päivän, ajoi toisen,                   \\
ajoi kohta kolmannenki. Jo päivänä kolmantena                 \\
päätyi Väinölän ahoille, Kalevalan kankahille.                \\
                                                              \\
Vaka vanha Väinämöinen, tietäjä iän-ikuinen,                  \\
oli teittensä ajaja, matkojensa mittelijä                     \\
noilla Väinölän ahoilla, Kalevalan kankahilla.                \\
                                                              \\
Tuli nuori Joukahainen, ajoi tiellä vastatusten:              \\
tarttui aisa aisan päähän, rahe rahkehen takistui,            \\
länget puuttui länkilöihin, vemmel vempelen nenähän.          \\
                                                              \\
Siitä siinä seisotahan, seisotahan, mietitähän...             \\
vesi vuoti vempelestä, usva aisoista usisi.                   \\
                                                              \\
Kysyi vanha Väinämöinen: "Kuit' olet sinä sukua,              \\
kun tulit tuhmasti etehen, vastahan varattomasti?             \\
Säret länget länkäpuiset, vesapuiset vempelehet,              \\
korjani pilastehiksi, rämäksi re'en retukan!"                 \\
                                                              \\
Silloin nuori Joukahainen sanan virkkoi, noin nimesi:         \\
"Mie olen nuori Joukahainen. Vaan sano oma sukusi:            \\
kuit' olet sinä sukua, kuta, kurja, joukkioa?"                \\
                                                              \\
Vaka vanha Väinämöinen jo tuossa nimittelihe.                 \\
Sai siitä sanoneheksi: "Kun liet nuori Joukahainen,           \\
veäite syrjähän vähäisen! Sie olet nuorempi minua."           \\
                                                              \\
Silloin nuori Joukahainen sanan virkkoi, noin nimesi:         \\
"Vähä on miehen nuoruuesta, nuoruuesta, vanhuuesta!           \\
                                                              \\
Kumpi on tieolta parempi, muistannalta mahtavampi,            \\
sep' on tiellä seisokahan, toinen tieltä siirtykähän.         \\
Lienet vanha Väinämöinen, laulaja iän-ikuinen,                \\
ruvetkamme laulamahan, saakamme sanelemahan,                  \\
mies on miestä oppimahan, toinen toista voittamahan!"         \\
                                                              \\
Vaka vanha Väinämöinen sanan virkkoi, noin nimesi:            \\
"Mitäpä minusta onpi laulajaksi, taitajaksi!                  \\
Ain' olen aikani elellyt näillä yksillä ahoilla,              \\
kotipellon pientarilla kuunnellut kotikäkeä.                  \\
Vaan kuitenki kaikitenki sano korvin kuullakseni:             \\
mitä sie enintä tieät, yli muien ymmärtelet?"                 \\
                                                              \\
Sanoi nuori Joukahainen: "Tieänpä minä jotaki!                \\
Sen on tieän selvällehen, tajuelen tarkoillehen:              \\
reppänä on liki lakea, liki lieska kiukoata.                  \\
                                                              \\
"Hyvä on hylkehen eleä, ve'en koiran viehkuroia:              \\
luotansa lohia syöpi, sivultansa siikasia.                    \\
                                                              \\
"Siiall' on sileät pellot, lohella laki tasainen.             \\
Hauki hallalla kutevi, kuolasuu kovalla säällä.               \\
Ahven arka, kyrmyniska sykysyt syvillä uipi,                  \\
kesät kuivilla kutevi, rantasilla rapsehtivi.                 \\
                                                              \\
"Kun ei tuosta kyllin liene, vielä tieän muunki tieon,        \\
arvoan yhen asian: pohjola porolla kynti,                     \\
etelä emähevolla, takalappi tarvahalla.                       \\
Tieän puut Pisan mäellä, hongat Hornan kalliolla:             \\
pitkät on puut Pisan mäellä, hongat Hornan kalliolla.         \\
                                                              \\
"Kolme on koskea kovoa, kolme järveä jaloa,                   \\
kolme vuorta korkeata tämän ilman kannen alla:                \\
Hämehess' on Hälläpyörä, Kaatrakoski Karjalassa;              \\
ei ole Vuoksen voittanutta, yli käynyttä Imatran."            \\
                                                              \\
Sanoi vanha Väinämöinen: "Lapsen tieto, naisen muisti,        \\
ei ole partasuun urohon eikä miehen naisekkahan!              \\
Sano syntyjä syviä, asioita ainoisia!"                        \\
                                                              \\
Se on nuori Joukahainen sanan virkkoi, noin nimesi:           \\
"Tieän mä tiaisen synnyn, tieän linnuksi tiaisen,             \\
kyyn viherän käärmeheksi, kiiskisen ve'en kalaksi.            \\
Rauan tieän raukeaksi, mustan mullan muikeaksi,               \\
varin veen on vaikeaksi, tulen polttaman pahaksi.             \\
                                                              \\
"Vesi on vanhin voitehista, kosken kuohu katsehista,          \\
itse Luoja loitsijoista, Jumala parantajista.                 \\
                                                              \\
"Vuoresta on vetosen synty, tulen synty taivosesta,           \\
alku rauan ruostehesta, vasken kanta kalliosta.               \\
                                                              \\
"Mätäs on märkä maita vanhin, paju puita ensimmäinen,         \\
hongan juuri huonehia, paatonen patarania."                   \\
                                                              \\
Vaka vanha Väinämöinen itse tuon sanoiksi virkki:             \\
"Muistatko mitä enemmin, vain jo loppuivat lorusi?"           \\
                                                              \\
Sanoi nuori Joukahainen: "Muistan vieläki vähäisen!           \\
Muistanpa ajan mokoman, kun olin merta kyntämässä,            \\
meren kolkot kuokkimassa, kalahauat kaivamassa,               \\
syänveet syventämässä, lampiveet on laskemassa,               \\
mäet mylleröittämässä, louhet luomassa kokohon.               \\
                                                              \\
"Viel' olin miesnä kuuentena, seitsemäntenä urosna            \\
tätä maata saataessa, ilmoa suettaessa,                       \\
ilman pieltä pistämässä, taivon kaarta kantamassa,            \\
kuuhutta kulettamassa, aurinkoa auttamassa,                   \\
otavaa ojentamassa, taivoa tähittämässä."                     \\
                                                              \\
Sanoi vanha Väinämöinen: "Sen varsin valehtelitki!            \\
Ei sinua silloin nähty, kun on merta kynnettihin,             \\
meren kolkot kuokittihin, kalahauat kaivettihin,              \\
syänveet syvennettihin, lampiveet on laskettihin,             \\
mäet mylleröitettihin, louhet luotihin kokohon.               \\
                                                              \\
"Eikä lie sinua nähty, ei lie nähty eikä kuultu               \\
tätä maata saataessa, ilmoa suettaessa,                       \\
ilman pieltä pistettäissä, taivon kaarta kannettaissa,        \\
kuuhutta kuletettaissa, aurinkoa autettaissa,                 \\
otavaa ojennettaissa, taivoa tähitettäissä."                  \\
                                                              \\
Se on nuori Joukahainen tuosta tuon sanoiksi virkki:          \\
"Kun ei lie minulla mieltä, kysyn mieltä miekaltani.          \\
Oi on vanha Väinämöinen, laulaja laveasuinen!                 \\
Lähe miekan mittelöhön, käypä kalvan katselohon!"             \\
                                                              \\
Sanoi vanha Väinämöinen: "En noita pahoin pelänne             \\
miekkojasi, mieliäsi, tuuriasi, tuumiasi.                     \\
Vaan kuitenki kaikitenki lähe en miekan mittelöhön            \\
sinun kanssasi, katala, kerallasi, kehno raukka."             \\
                                                              \\
Siinä nuori Joukahainen murti suuta, väänti päätä,            \\
murti mustoa haventa. Itse tuon sanoiksi virkki:              \\
"Ken ei käy miekan mittelöhön, lähe ei kalvan                 \\
katselohon,                                                   \\
sen minä siaksi laulan, alakärsäksi asetan.                   \\
Panen semmoiset urohot sen sikäli, tuon täkäli,               \\
sorran sontatunkiohon, läävän nurkkahan nutistan."            \\
                                                              \\
Siitä suuttui Väinämöinen, siitä suuttui ja häpesi.           \\
Itse loihe laulamahan, sai itse sanelemahan:                  \\
ei ole laulut lasten laulut, lasten laulut, naisten naurut,   \\
ne on partasuun urohon, joit' ei laula kaikki lapset          \\
eikä pojat puoletkana, kolmannetkana kosijat                  \\
tällä inhalla iällä, katovalla kannikalla.                    \\
                                                              \\
Lauloi vanha Väinämöinen: järvet läikkyi, maa järisi,         \\
vuoret vaskiset vapisi, paaet vahvat paukahteli,              \\
kalliot kaheksi lenti, kivet rannoilla rakoili.               \\
                                                              \\
Lauloi nuoren Joukahaisen: vesat lauloi vempelehen,           \\
pajupehkon länkilöihin, raiat rahkehen nenähän.               \\
Lauloi korjan kultalaian: lauloi lampihin haoiksi;            \\
lauloi ruoskan helmiletkun meren rantaruokosiksi;             \\
lauloi laukkipään hevosen kosken rannalle kiviksi.            \\
                                                              \\
Lauloi miekan kultakahvan salamoiksi taivahalle,              \\
siitä jousen kirjavarren kaariksi vesien päälle,              \\
siitä nuolensa sulitut havukoiksi kiitäviksi,                 \\
siitä koiran koukkuleuan, sen on maahan maakiviksi.           \\
                                                              \\
Lakin lauloi miehen päästä pilven pystypää kokaksi;           \\
lauloi kintahat käestä umpilammin lumpehiksi,                 \\
siitä haljakan sinisen hattaroiksi taivahalle,                \\
vyöltä ussakan utuisen halki taivahan tähiksi.                \\
                                                              \\
Itsen lauloi Joukahaisen: lauloi suohon suonivöistä,          \\
niittyhyn nivuslihoista, kankahasen kainaloista.              \\
                                                              \\
Jo nyt nuori Joukahainen jopa tiesi jotta tunsi:              \\
tiesi tielle tullehensa, matkallen osannehensa                \\
voittelohon, laulelohon kera vanhan Väinämöisen.              \\
                                                              \\
Jaksoitteli jalkoansa: eipä jaksa jalka nousta;               \\
toki toistakin yritti: siin' oli kivinen kenkä.               \\
                                                              \\
Siitä nuoren Joukahaisen jopa tuskaksi tulevi,                \\
läylemmäksi lankeavi. Sanan virkkoi, noin nimesi:             \\
"Oi on viisas Väinämöinen, tietäjä iän-ikuinen!               \\
Pyörrytä pyhät sanasi, peräytä lausehesi!                     \\
Päästä tästä pälkähästä, tästä seikasta selitä!               \\
Panenpa parahan makson, annan lunnahat lujimmat."             \\
                                                              \\
Sanoi vanha Väinämöinen: "Niin mitä minullen annat,           \\
jos pyörrän pyhät sanani, peräytän lauseheni,                 \\
päästän siitä pälkähästä, siitä seikasta selitän?"            \\
                                                              \\
Sanoi nuori Joukahainen: "Onp' on mulla kaarta kaksi,         \\
jousta kaksi kaunokaista; yks' on lyömähän riveä,             \\
toinen tarkka ammunnalle. Ota niistä jompikumpi!"             \\
                                                              \\
Sanoi vanha Väinämöinen: "Huoli en, hurja, jousistasi,        \\
en, katala, kaaristasi! On noita itselläniki                  \\
joka seinä seisoteltu, joka vaarnanen varottu:                \\
miehittä metsässä käyvät, urohitta ulkotöillä."               \\
Lauloi nuoren Joukahaisen, lauloi siitäki syvemmä.            \\
                                                              \\
Sanoi nuori Joukahainen: "Onp' on mulla purtta kaksi,         \\
kaksi kaunoista venoa; yks' on kiistassa kepeä,               \\
toinen paljo kannattava. Ota niistä jompikumpi!"              \\
                                                              \\
Sanoi vanha Väinämöinen: "Enp' on huoli pursistasi,           \\
venehistäsi valita! On noita itselläniki                      \\
joka tela tempaeltu, joka lahtema laottu,                     \\
mikä tuulella tukeva, mikä vastasään menijä."                 \\
Lauloi nuoren Joukahaisen, lauloi siitäki syvemmä.            \\
                                                              \\
Sanoi nuori Joukahainen: "On mulla oritta kaksi,              \\
kaksi kaunoista hepoa; yks' on juoksulle jalompi,             \\
toinen raisu rahkehille. Ota niistä jompikumpi!"              \\
                                                              \\
Sanoi vanha Väinämöinen: "En huoli hevosiasi,                 \\
sure en sukkajalkojasi! On noita itselläniki                  \\
joka soimi solmieltu, joka tanhua taluttu:                    \\
vesi selvä selkäluilla, rasvalampi lautasilla."               \\
Lauloi nuoren Joukahaisen, lauloi siitäki syvemmä.            \\
                                                              \\
Sanoi nuori Joukahainen: "Oi on vanha Väinämöinen!            \\
Pyörrytä pyhät sanasi, peräytä lausehesi!                     \\
Annan kultia kypärin, hope'ita huovan täyen,                  \\
isoni soasta saamat, taluttamat tappelosta."                  \\
                                                              \\
Sanoi vanha Väinämöinen: "En huoli hope'itasi,                \\
kysy en, kurja, kultiasi! On noita itselläniki                \\
joka aitta ahtaeltu, joka vakkanen varottu:                   \\
ne on kullat kuun-ikuiset, päivän-polviset hopeat."           \\
Lauloi nuoren Joukahaisen, lauloi siitäki syvemmä.            \\
                                                              \\
Sanoi nuori Joukahainen: "Oi on vanha Väinämöinen!            \\
Päästä tästä pälkähästä, tästä seikasta selitä!               \\
Annan aumani kotoiset, heitän hietapeltoseni                  \\
oman pääni päästimeksi, itseni lunastimeksi."                 \\
                                                              \\
Sanoi vanha Väinämöinen: "En halaja aumojasi,                 \\
herjä, hietapeltojasi! On noita itselläniki,                  \\
peltoja joka perällä, aumoja joka aholla.                     \\
Omat on paremmat pellot, omat aumat armahammat."              \\
Lauloi nuoren Joukahaisen, lauloi ainakin alemma.             \\
                                                              \\
Siitä nuori Joukahainen toki viimein tuskastui,               \\
kun oli leuan liettehessä, parran paikassa pahassa,           \\
suun on suossa, sammalissa, hampahin haon perässä.            \\
                                                              \\
Sanoi nuori Joukahainen: "Oi on viisas Väinämöinen,           \\
tietäjä iän-ikuinen! Laula jo laulusi takaisin,               \\
heitä vielä heikko henki, laske täältä pois minua!            \\
Virta jo jalkoa vetävi, hiekka silmiä hiovi.                  \\
                                                              \\
"Kun pyörrät pyhät sanasi, luovuttelet luottehesi,            \\
annan Aino siskoseni, lainoan emoni lapsen                    \\
sulle pirtin pyyhkijäksi, lattian lakaisijaksi,               \\
hulikkojen huuhtojaksi, vaippojen viruttajaksi,               \\
kutojaksi kultavaipan, mesileivän leipojaksi."                \\
                                                              \\
Siitä vanha Väinämöinen ihastui ikihyväksi,                   \\
kun sai neion Joukahaisen vanhan päivänsä varaksi.            \\
                                                              \\
Istuiksen ilokivelle, laulupaaelle paneikse.                  \\
Lauloi kotvan, lauloi toisen, lauloi kotvan kolmannenki:      \\
pyörti pois pyhät sanansa, perin laski lausehensa.            \\
                                                              \\
Pääsi nuori Joukahainen, pääsi leuan liettehestä,             \\
parran paikasta pahasta, hevonen kosken kivestä,              \\
reki rannalta haosta, ruoska rannan ruokosesta.               \\
                                                              \\
Kohoeli korjahansa, reutoihe rekosehensa;                     \\
läksi mielellä pahalla, syämellä synkeällä                    \\
luoksi armahan emonsa, tykö valtavanhempansa.                 \\
                                                              \\
Ajoa karittelevi. Ajoi kummasti kotihin:                      \\
rikki riihe'en rekensä, aisat poikki portahasen.              \\
                                                              \\
Alkoi äiti arvaella, isonen sanan sanovi:                     \\
"Suottapa rikoit rekesi, tahallasi aisan taitoit!             \\
Mitäpä kummasti kuletki, tulet tuhmasti kotihin?"             \\
                                                              \\
Tuossa nuori Joukahainen itkeä vetistelevi                    \\
alla päin, pahoilla mielin, kaiken kallella kypärin           \\
sekä huulin hyypynyisin, nenän suulle langennuisen.           \\
                                                              \\
Emo ennätti kysyä, vaivan nähnyt vaaitella:                   \\
"Mitä itket, poikueni, nuorna saamani, nureksit,              \\
olet huulin hyypynyisin, nenän suulle langennuisen?"          \\
                                                              \\
Sanoi nuori Joukahainen: "Oi on maammo, kantajani!            \\
Jo on syytä syntynynnä, taikoja tapahtununna,                 \\
syytä kyllin itkeäni, taikoja nureksiani!                     \\
Tuot' itken tämän ikäni, puhki polveni murehin:               \\
annoin Aino siskoseni, lupasin emoni lapsen                   \\
Väinämöiselle varaksi, laulajalle puolisoksi,                 \\
turvaksi tutisevalle, suojaksi sopenkululle."                 \\
                                                              \\
Emo kahta kämmentänsä hykersi molempiansa;                    \\
sanan virkkoi, noin nimesi: "Elä itke, poikueni!              \\
Ei ole itkettäviä, suuresti surettavia:                       \\
tuota toivoin tuon ikäni, puhki polveni halasin               \\
sukuhuni suurta miestä, rotuhuni rohkeata,                    \\
vävykseni Väinämöistä, laulajata langokseni."                 \\
                                                              \\
Sisar nuoren Joukahaisen itse itkullen apeutui.               \\
Itki päivän, itki toisen poikkipuolin portahalla;             \\
itki suuresta surusta, apeasta miel'alasta.                   \\
                                                              \\
Sai emo sanelemahan: "Mitä itket, Ainoseni,                   \\
kun olet saava suuren sulhon, miehen korkean kotihin          \\
ikkunoillen istujaksi, lautsoille lavertajaksi?"              \\
                                                              \\
Tuon tytär sanoiksi virkki: "Oi emoni, kantajani!             \\
Itkenpä minä jotaki: itken kassan kauneutta,                  \\
tukan nuoren tuuheutta, hivuksien hienoutta,                  \\
jos ne piennä peitetähän, katetahan kasvavana.                \\
                                                              \\
"Tuotapa ikäni itken, tuota päivän armautta,                  \\
suloutta kuun komean, ihanuutta ilman kaiken,                 \\
jos oisi nuorna jättäminen, lapsena unohtaminen               \\
veikon veistotanterille, ison ikkunan aloille."               \\
                                                              \\
Sanovi emo tytölle, lausui vanhin lapsellensa:                \\
"Mene, huima, huolinesi, epäkelpo, itkuinesi!                 \\
Ei ole syytä synkistyä, aihetta apeutua.                      \\
Paistavi Jumalan päivä muuallaki maailmassa,                  \\
ei isosi ikkunoilla, veikkosi veräjän suulla.                 \\
Myös on marjoja mäellä, ahomailla mansikoita                  \\
poimia sinun poloisen ilmassa etempänäki,                     \\
ei aina ison ahoilla, veikon viertokankahilla."               \\