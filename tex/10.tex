% Kymmenes Runo

\chapter*{Kymmenes: Runo}

\colorini{V}{aka vanha Väinämöinen} otti ruskean orihin,          \\
pani varsan valjahisin, ruskean re'en etehen;                     \\
itse reuoikse rekehen, kohennaikse korjahansa.                    \\
                                                                  \\
Laski virkkua vitsalla, helähytti helmisvyöllä;                   \\
virkku juoksi, matka joutui, reki vieri, tie lyheni,              \\
jalas koivuinen kolasi, vemmel piukki pihlajainen.                \\
                                                                  \\
Ajavi karettelevi. Ajoi soita, ajoi maita,                        \\
ajoi aavoja ahoja. Kulki päivän, kulki toisen,                    \\
niin päivällä kolmannella tuli pitkän sillan päähän,              \\
Kalevalan kankahalle, Osmon pellon pientarelle.                   \\
                                                                  \\
Siinä tuon sanoiksi virkki, itse lausui ja pakisi:                \\
"Syö, susi, unennäkijä, tapa, tauti, lappalainen!                 \\
Sanoi ei saavani kotihin enämpi elävin silmin                     \\
sinä ilmoisna ikänä, kuuna kullan valkeana                        \\
näille Väinölän ahoille, Kalevalan kankahille."                   \\
                                                                  \\
Siitä vanha Väinämöinen laulelevi, taitelevi:                     \\
lauloi kuusen kukkalatvan, kukkalatvan, kultalehvän;              \\
latvan työnti taivahalle, puhki pilvien kohotti,                  \\
lehvät ilmoille levitti, halki taivahan hajotti.                  \\
                                                                  \\
Laulelevi, taitelevi: lauloi kuun kumottamahan                    \\
kultalatva-kuusosehen, lauloi oksillen otavan.                    \\
                                                                  \\
Ajavi karettelevi kohti kullaista kotia,                          \\
alla päin, pahoilla mielin, kaiken kallella kypärin,              \\
kun oli seppo Ilmarisen, takojan iän-ikuisen,                     \\
luvannut lunastimeksi, oman päänsä päästimeksi                    \\
pimeähän Pohjolahan, summahan Sariolahan.                         \\
                                                                  \\
Jop' on seisottui oronen Osmon uuen pellon päähän.                \\
Siitä vanha Väinämöinen päätä korjasta kohotti:                   \\
kuuluvi pajasta pauke, hilke hiilihuonehesta.                     \\
                                                                  \\
Vaka vanha Väinämöinen itse pistihe pajahan.                      \\
Siell' on seppo Ilmarinen: takoa taputtelevi.                     \\
Sanoi seppo Ilmarinen: "Oi sie vanha Väinämöinen!                 \\
Miss' olet viikon viipynynnä, kaiken aikasi asunut?"              \\
                                                                  \\
Vaka vanha Väinämöinen itse tuon sanoiksi virkki:                 \\
"Tuoll' olen viikon viipynynnä, kaiken aikani elellyt             \\
pimeässä Pohjolassa, summassa Sariolassa,                         \\
liukunut Lapin lauilla, tietomiesten tienohilla."                 \\
                                                                  \\
Siitä seppo Ilmarinen sanan virkkoi, noin nimesi:                 \\
"Oi sie vanha Väinämöinen, tietäjä iän-ikuinen!                   \\
Mitä lausut matkoiltasi tultua kotituville?"                      \\
                                                                  \\
Virkki vanha Väinämöinen: "Äijä on mulla lausumista:              \\
onp' on neiti Pohjolassa, impi kylmässä kylässä,                  \\
jok' ei suostu sulhosihin, mielly miehi'in hyvihin.               \\
Kiitti puoli Pohjan maata, kun onpi kovin korea:                  \\
kuuhut paistoi kulmaluilta, päivä rinnoilta risotti,              \\
otavainen olkapäiltä, seitsentähtinen selältä.                    \\
                                                                  \\
"Sinä, seppo Ilmarinen, takoja iän-ikuinen,                       \\
lähe neittä noutamahan, päätä kassa katsomahan!                   \\
Kun saatat takoa sammon, kirjokannen kirjaella,                   \\
niin saat neion palkastasi, työstäsi tytön ihanan."               \\
                                                                  \\
Sanoi seppo Ilmarinen: "Ohoh vanha Väinämöinen!                   \\
Joko sie minut lupasit pimeähän Pohjolahan                        \\
oman pääsi päästimeksi, itsesi lunastimeksi?                      \\
En sinä pitkänä ikänä, kuuna kullan valkeana                      \\
lähe Pohjolan tuville, Sariolan salvoksille,                      \\
miesten syöjille sijoille, urosten upottajille."                  \\
                                                                  \\
Siitä vanha Väinämöinen itse tuon sanoiksi virkki:                \\
"Viel' on kumma toinen kumma: onp' on kuusi                       \\
kukkalatva,                                                       \\
kukkalatva, kultalehvä, Osmon pellon pientarella;                 \\
kuuhut latvassa kumotti, oksilla otava seisoi."                   \\
                                                                  \\
Sanoi seppo Ilmarinen: "En usko toeksi tuota,                     \\
kun en käyne katsomahan, nähne näillä silmilläni."                \\
Sanoi vanha Väinämöinen: "Kun et usko kuitenkana,                 \\
lähtekämme katsomahan, onko totta vai valetta!"                   \\
                                                                  \\
Lähettihin katsomahan tuota kuusta kukkapäätä,                    \\
yksi vanha Väinämöinen, toinen seppo Ilmarinen.                   \\
Sitte tuonne tultuansa Osmon pellon pientarelle                   \\
seppo seisovi likellä, uutta kuusta kummeksivi,                   \\
kun oli oksilla otava, kuuhut kuusen latvasessa.                  \\
                                                                  \\
Siinä vanha Väinämöinen itse tuon sanoiksi virkki:                \\
"Nyt sinä, seppo veikkoseni, nouse kuuta noutamahan,              \\
otavaista ottamahan kultalatva-kuusosesta!"                       \\
                                                                  \\
Siitä seppo Ilmarinen nousi puuhun korkealle,                     \\
ylähäksi taivahalle, nousi kuuta noutamahan,                      \\
otavaista ottamahan kultalatva-kuusosesta.                        \\
                                                                  \\
Virkki kuusi kukkalatva, lausui lakkapää petäjä:                  \\
"Voipa miestä mieletöintä, äkkioutoa urosta!                      \\
Nousit, outo, oksilleni, lapsen-mieli, latvahani                  \\
kuvakuun on nouantahan, valetähtyen varahan!"                     \\
                                                                  \\
Silloin vanha Väinämöinen lauloa hyrähtelevi:                     \\
lauloi tuulen tuppurihin, ilman raivohon rakenti;                 \\
sanovi sanalla tuolla, lausui tuolla lausehella:                  \\
"Ota, tuuli, purtehesi, ahava, venosehesi                         \\
vieä vieretelläksesi pimeähän Pohjolahan!"                        \\
                                                                  \\
Nousi tuuli tuppurihin, ilma raivohon rakentui,                   \\
otti seppo Ilmarisen vieä viiletelläksensä                        \\
pimeähän Pohjolahan, summahan Sariolahan.                         \\
                                                                  \\
Siinä seppo Ilmarinen jopa kulki jotta joutui!                    \\
Kulki tuulen tietä myöten, ahavan ratoa myöten,                   \\
yli kuun, alatse päivän, otavaisten olkapäitse;                   \\
päätyi Pohjolan pihalle, Sariolan saunatielle,                    \\
eikä häntä koirat kuullut eikä haukkujat havainnut.               \\
                                                                  \\
Louhi, Pohjolan emäntä, Pohjan akka harvahammas                   \\
tuop' on päätyvi pihalle. Itse ennätti sanoa:                     \\
"Mi sinä lienet miehiäsi ja kuka urohiasi?                        \\
Tulit tänne tuulen tietä, ahavan rekiratoa,                       \\
eikä koirat kohti hauku, villahännät virkkaele!"                  \\
                                                                  \\
Sanoi seppo Ilmarinen: "En mä tänne tullutkana                    \\
kylän koirien kuluiksi, villahäntien vihoiksi,                    \\
näillen ouoillen oville, veräjille vierahille."                   \\
                                                                  \\
Siitä Pohjolan emäntä tutkaeli tullehelta:                        \\
"Oletko tullut tuntemahan, kuulemahan, tietämähän                 \\
tuota seppo Ilmarista, takojata taitavinta?                       \\
Jo on viikon vuotettuna sekä kauan kaivattuna                     \\
näille Pohjolan perille uuen sammon laaintahan."                  \\
                                                                  \\
Se on seppo Ilmarinen sanan virkkoi, noin nimesi:                 \\
"Lienen tullut tuntemahan tuon on seppo Ilmarisen,                \\
kun olen itse Ilmarinen, itse taitava takoja."                    \\
                                                                  \\
Louhi, Pohjolan emäntä, Pohjan akka harvahammas,                  \\
pian pistihe tupahan, sanovi sanalla tuolla:                      \\
"Neityeni nuorempani, lapseni vakavimpani!                        \\
Pane nyt päällesi parasta, varrellesi valke'inta,                 \\
hempe'intä helmoillesi, ripe'intä rinnoillesi,                    \\
kaulallesi kaunihinta, kukke'inta kulmillesi,                     \\
poskesi punottamahan, näköpääsi näyttämähän!                      \\
Jo on seppo Ilmarinen, takoja iän-ikuinen,                        \\
saanut sammon laaintahan, kirjokannen kirjantahan."               \\
                                                                  \\
Tuop' on kaunis Pohjan tytti, maan kuulu, ve'en valio,            \\
otti vaattehet valitut, pukehensa puhtahimmat;                    \\
viitiseikse, vaatiseikse, pääsomihin suoritseikse,                \\
vaskipantoihin paneikse, kultavöihin kummitseikse.                \\
                                                                  \\
Tuli aitasta tupahan, kaapsahellen kartanolta                     \\
silmistänsä sirkeänä, korvistansa korkeana,                       \\
kaunihina kasvoiltansa, poskilta punehtivana;                     \\
kullat riippui rinnan päällä, pään päällä hopeat huohti.          \\
                                                                  \\
Itse Pohjolan emäntä käytti seppo Ilmarisen                       \\
noissa Pohjan tuvissa, Sariolan salvoksissa;                      \\
siellä syötti syöneheksi, juotti miehen juoneheksi,               \\
apatti ani hyväksi. Sai tuosta sanelemahan:                       \\
                                                                  \\
"Ohoh seppo Ilmarinen, takoja iän-ikuinen!                        \\
Saatatko takoa sammon, kirjokannen kirjaella                      \\
joutsenen kynän nenästä, maholehmän maitosesta,                   \\
ohran pienestä jyvästä, kesäuuhen untuvasta,                      \\
niin saat neion palkastasi, työstäsi tytön ihanan."               \\
                                                                  \\
Silloin seppo Ilmarinen itse tuon sanoiksi virkki:                \\
"Saattanen takoa sammon, kirjokannen kalkutella                   \\
joutsenen kynän nenästä, maholehmän maitosesta,                   \\
ohran pienestä jyvästä, kesäuuhen untuvasta,                      \\
kun olen taivoa takonut, ilman kantta kalkuttanut                 \\
ilman alkusen alutta, riporihman tehtyisettä."                    \\
                                                                  \\
Läksi sammon laaintahan, kirjokannen kirjontahan.                 \\
Kysyi paikalta pajoa, kaipasi sepinkaluja:                        \\
ei ole paikalla pajoa, ei pajoa, ei paletta,                      \\
ahjoa, alasintana, vasarata, varttakana!                          \\
                                                                  \\
Silloin seppo Ilmarinen sanan virkkoi, noin nimesi:               \\
"Akatp' on epäelköhöt, herjat kesken heittäköhöt,                 \\
eip' on mies pahempikana, uros untelompikana!"                    \\
                                                                  \\
Etsi ahjollen alusta, leveyttä lietsehelle                        \\
noilla mailla, mantereilla, Pohjan peltojen perillä.              \\
                                                                  \\
Etsi päivän, etsi toisen. Jo päivänä kolmantena                   \\
tuli kirjava kivonen, vahatukko vastahansa.                       \\
Tuohon seppo seisottihe, takoja tulen rakenti;                    \\
päivän laati palkehia, toisen ahjoa asetti.                       \\
                                                                  \\
Siitä seppo Ilmarinen, takoja iän-ikuinen,                        \\
tunki ainehet tulehen, takehensa alle ahjon;                      \\
otti orjat lietsomahan, väkipuolet vääntämähän.                   \\
                                                                  \\
Orjat lietsoi löyhytteli, väkipuolet väännätteli                  \\
kolme päiveä kesäistä ja kolme kesäistä yötä:                     \\
kivet kasvoi kantapäihin, vahat varvasten sijoille.               \\
                                                                  \\
Niin päivänä ensimäisnä itse seppo Ilmarinen                      \\
kallistihe katsomahan ahjonsa alaista puolta,                     \\
mitä tullehe tulesta, selvinnehe valkeasta.                       \\
Jousi tungeikse tulesta, kaasi kulta kuumoksesta,                 \\
kaari kulta, pää hopea, varsi vasken-kirjavainen.                 \\
                                                                  \\
On jousi hyvän näköinen, vaan onpi pahan tapainen:                \\
joka päivä pään kysyvi, parahana kaksi päätä.                     \\
                                                                  \\
Itse seppo Ilmarinen ei tuota kovin ihastu:                       \\
kaaren katkaisi kaheksi, siitä tunkevi tulehen;                   \\
laitti orjat lietsomahan, väkipuolet vääntämähän.                 \\
                                                                  \\
Jop' on päivänä jälestä itse seppo Ilmarinen                      \\
kallistihe katsomahan ahjonsa alaista puolta:                     \\
veno tungeikse tulesta, punapursi kuumoksesta,                    \\
kokat kullan kirjaeltu, hangat vaskesta valettu.                  \\
                                                                  \\
On veno hyvän näköinen, ei ole hyvän tapainen:                    \\
suotta lähtisi sotahan, tarpehetta tappelohon.                    \\
                                                                  \\
Se on seppo Ilmarinen ei ihastu tuotakana:                        \\
venon murskaksi murenti, tunkevi tulisijahan;                     \\
laitti orjat lietsomahan, väkipuolet vääntämähän.                 \\
                                                                  \\
Jo päivänä kolmantena itse seppo Ilmarinen                        \\
kallistihe katsomahan ahjonsa alaista puolta:                     \\
hieho tungeikse tulesta, sarvi kulta kuumoksesta,                 \\
otsassa otavan tähti, päässä päivän pyöryläinen.                  \\
                                                                  \\
On hieho hyvän näköinen, ei ole hyvän tapainen:                   \\
metsässä makaelevi, maion maahan kaatelevi.                       \\
                                                                  \\
Se on seppo Ilmarinen ei ihastu tuotakana:                        \\
lehmän leikkeli paloiksi, siitä tunkevi tulehen;                  \\
laitti orjat lietsomahan, väkipuolet vääntämähän.                 \\
                                                                  \\
Jo päivänä neljäntenä itse seppo Ilmarinen                        \\
kallistihe katsomahan ahjonsa alaista puolta:                     \\
aura tungeikse tulesta, terä kulta kuumoksesta,                   \\
terä kulta, vaski varsi, hopeata ponnen päässä.                   \\
                                                                  \\
On aura hyvän näköinen, ei ole hyvän tapainen:                    \\
kylän pellot kyntelevi, vainiot vakoelevi.                        \\
                                                                  \\
Se on seppo Ilmarinen ei ihastu tuotakana:                        \\
auran katkaisi kaheksi, alle ahjonsa ajavi.                       \\
                                                                  \\
Laittoi tuulet lietsomahan, väkipuuskat vääntämähän.              \\
                                                                  \\
Lietsoi tuulet löyhytteli: itä lietsoi, lietsoi länsi,            \\
etelä enemmän lietsoi, pohjanen kovin porotti.                    \\
Lietsoi päivän, lietsoi toisen, lietsoi kohta kolmannenki:        \\
tuli tuiski ikkunasta, säkehet ovesta säykkyi,                    \\
tomu nousi taivahalle, savu pilvihin sakeni.                      \\
                                                                  \\
Se on seppo Ilmarinen päivän kolmannen perästä                    \\
kallistihe katsomahan ahjonsa alaista puolta:                     \\
näki sammon syntyväksi, kirjokannen kasvavaksi.                   \\
                                                                  \\
Siitä seppo Ilmarinen, takoja iän-ikuinen,                        \\
takoa taputtelevi, lyöä lynnähyttelevi.                           \\
Takoi sammon taitavasti: laitahan on jauhomyllyn,                 \\
toisehen on suolamyllyn, rahamyllyn kolmantehen.                  \\
                                                                  \\
Siitä jauhoi uusi sampo, kirjokansi kiikutteli,                   \\
jauhoi purnun puhtehessa: yhen purnun syötäviä,                   \\
toisen jauhoi myötäviä, kolmannen kotipitoja.                     \\
                                                                  \\
Niin ihastui Pohjan akka; saattoi sitte sammon suuren             \\
Pohjolan kivimäkehen, vaaran vaskisen sisähän                     \\
yheksän lukon ta'aksi. Siihen juuret juuttutteli                  \\
yheksän sylen syvähän: juuren juurti maaemähän,                   \\
toisen vesiviertehesen, kolmannen kotimäkehen.                    \\
                                                                  \\
Siitä seppo Ilmarinen tyttöä anelemahan.                          \\
Sanan virkkoi, noin nimesi: "Joko nyt minulle neiti,              \\
kun sai sampo valmihiksi, kirjokansi kaunihiksi?"                 \\
                                                                  \\
Tuop' on kaunis Pohjan tyttö itse noin sanoiksi virkki:           \\
"Kukapa tässä toisna vuonna, kenpä kolmanna kesänä                \\
käkiä kukutteleisi, lintusia laulattaisi,                         \\
jos minä menisin muunne, saisin, marja, muille maille!            \\
                                                                  \\
"Jos tämä kana katoisi, tämä hanhi hairahtaisi,                   \\
eksyisi emosen tuoma, punapuola pois menisi,                      \\
kaikkipa käet katoisi, ilolinnut liikahtaisi                      \\
tämän kunnahan kukuilta, tämän harjun hartehilta.                 \\
                                                                  \\
"Enkä joua ilmankana, pääse en neitipäiviltäni,                   \\
noilta töiltä tehtäviltä, kesäisiltä kiirehiltä:                  \\
marjat on maalla poimimatta, lahen rannat laulamatta,             \\
astumattani ahoset, lehot leikin lyömättäni."                     \\
                                                                  \\
Siitä seppo Ilmarinen, takoja iän-ikuinen,                        \\
alla päin, pahoilla mielin, kaiken kallella kypärin               \\
jo tuossa ajattelevi, pitkin päätänsä pitävi,                     \\
miten kulkea kotihin, tulla maille tuttaville                     \\
pimeästä Pohjolasta, summasta Sariolasta.                         \\
                                                                  \\
Sanoi Pohjolan emäntä: "Ohoh seppo Ilmarinen!                     \\
Mit' olen pahoilla mielin, kaiken kallella kypärin?               \\
Laatisiko mieli mennä elomaillen entisille?"                      \\
                                                                  \\
Sanoi seppo Ilmarinen: "Sinne mieleni tekisi                      \\
kotihini kuolemahan, maalleni masenemahan."                       \\
                                                                  \\
Siitä Pohjolan emäntä syötti miehen, juotti miehen,               \\
istutti perähän purren melan vaskisen varahan;                    \\
virkki tuulen tuulemahan, pohjasen puhaltamahan.                  \\
                                                                  \\
Siitä seppo Ilmarinen, takoja iän-ikuinen,                        \\
matkasi omille maille ylitse meren sinisen.                       \\
Kulki päivän, kulki toisen; päivälläpä kolmannella                \\
jo tuli kotihin seppo, noille syntymäsijoille.                    \\
                                                                  \\
Kysyi vanha Väinämöinen Ilmariselta sepolta:                      \\
"Veli, seppo Ilmarinen, takoja iän-ikuinen!                       \\
Joko laait uuen sammon, kirjokannen kirjaelit?"                   \\
                                                                  \\
Sanoi seppo Ilmarinen, itse laatia pakisi:                        \\
"Jopa jauhoi uusi sampo, kirjokansi kiikutteli,                   \\
jauhoi purnun puhtehessa: yhen purnun syötäviä,                   \\
toisen jauhoi myötäviä, kolmannen pi'eltäviä."                    \\