% Seitsemäs runo

\chapter*{Seitsemäs: Runo}

\colorini{V}aka vanha Väinämöinen uipi aavoja syviä;      \\
kulki kuusisna hakona, petäjäisnä pehkiönä                \\
kuusi päiveä kesäistä, kuusi yötä järkiähän,              \\
eessänsä vesi vetelä, takanansa taivas selvä.             \\
                                                          \\
Uip' on vielä yötä kaksi, kaksi päiveä pisintä.           \\
Niin yönä yheksäntenä, kaheksannen päivän päästä          \\
toki tuskaksi tulevi, painuvi pakolliseksi,               \\
kun ei ole kynttä varpahissa eikä sormissa niveltä.       \\
                                                          \\
Siinä vanha Väinämöinen itse tuon sanoiksi virkki:        \\
"Voi minä poloinen poika, voi poika polon-alainen,        \\
kun läksin omilta mailta, elomailta entisiltä             \\
iäkseni ilman alle, kuuksi päiväksi kululle,              \\
tuulten tuuiteltavaksi, aaltojen ajeltavaksi              \\
näillä väljillä vesillä, ulapoilla auke'illa!             \\
Vilu on täällä ollakseni, vaiva värjätelläkseni,          \\
aina aalloissa asua, veen selällä seurustella.            \\
                                                          \\
"Enkä tuota tieäkänä, miten olla, kuin eleä               \\
tällä inhalla iällä, katovalla kannikalla:                \\
tuulehenko teen tupani, vetehenkö pirtin veistän?         \\
                                                          \\
"Teen mä tuulehen tupani: ei ole tuulessa tukea;          \\
veistän pirttini vetehen: vesi viepi veistokseni."        \\
                                                          \\
Lenti lintunen Lapista, kokkolintu koillisesta.           \\
Ei ole kokko suuren suuri eikä kokko pienen pieni:        \\
yksi siipi vettä viisti, toinen taivasta lakaisi,         \\
pursto merta pyyhätteli, nokka luotoja lotaisi.           \\
                                                          \\
Lenteleikse, liiteleikse, katseleikse, käänteleikse.      \\
Näki vanhan Väinämöisen selällä meren sinisen:            \\
"Mit' olet meressä, miesi, uros, aaltojen seassa?"        \\
                                                          \\
Vaka vanha Väinämöinen sanan virkkoi, noin nimesi:        \\
"Sit' olen meressä miesi, uros aaltojen varassa:          \\
läksin neittä Pohjolasta, impeä Pimentolasta.             \\
                                                          \\
"Ajoa karautimme suloa meryttä myöten.                    \\
Niin päivänä muutamana, huomenna moniahana                \\
tulin Luotolan lahelle, Joukolan jokivesille:             \\
hepo alta ammuttihin, itseäni mielittihin.                \\
                                                          \\
"Siitä vierähin vetehen, sorruin sormin lainehesen        \\
tuulen tuuiteltavaksi, aaltojen ajeltavaksi.              \\
                                                          \\
"Tulipa tuuli luotehesta, iästä iso vihuri;               \\
se mun kauas kannatteli, uitteli ulomma maasta.           \\
Mont' olen päiveä pälynnyt, monta yötä uiksennellut       \\
näitä väljiä vesiä, ulapoita auke'ita;                    \\
enk' on tuota tunnekana, arvoa, älyäkänä,                 \\
kumpi kuoloksi tulevi, kumpi ennen ennättävi:             \\
nälkähänkö nääntyminen, vai vetehen vaipuminen."          \\
                                                          \\
Sanoi kokko, ilman lintu: "Ellös olko milläskänä!         \\
Seisotaite selkähäni, nouse kynkkäluun nenille!           \\
Mie sinun merestä kannan, minne mielesi tekevi.           \\
Vielä muistan muunki päivän, arvoan ajan paremman,        \\
kun ajoit Kalevan kasken, Osmolan salon sivallit:         \\
heitit koivun kasvamahan, puun sorean seisomahan          \\
linnuille lepeämiksi, itselleni istumiksi."               \\
                                                          \\
Siitä vanha Väinämöinen kohottavi kokkoansa;              \\
mies on nousevi merestä, uros aallosta ajaikse,           \\
siiville sijoitteleikse, kokon kynkkäluun nenille.        \\
                                                          \\
Tuop' on kokko, ilman lintu, kantoi vanhan Väinämöisen,   \\
viepi tuulen tietä myöten, ahavan ratoa myöten            \\
Pohjan pitkähän perähän, summahan Sariolahan.             \\
Siihen heitti Väinämöisen, itse ilmahan kohosi.           \\
                                                          \\
Siinä itki Väinämöinen, siinä itki ja urisi               \\
rannalla merellisellä, nimen tietämättömällä,             \\
sata haavoa sivulla, tuhat tuulen pieksemätä,             \\
partaki pahoin kulunut, tukka mennyt tuuhakaksi.          \\
                                                          \\
Itki yötä kaksi, kolme, saman verran päiviäki;            \\
eikä tiennyt tietä käyä, outo, matkoa osannut             \\
palataksensa kotihin, mennä maille tuttaville,            \\
noille syntymäsijoille, elomaillen entisille.             \\
                                                          \\
Pohjan piika pikkarainen, vaimo valkeanverinen,           \\
teki liiton päivän kanssa, päivän kanssa, kuun keralla    \\
yhen ajan noustaksensa ja yhen havataksensa:              \\
itse ennen ennätteli, ennen kuuta, aurinkoa,              \\
kukonki kurahtamatta, kanan lapsen laulamatta.            \\
                                                          \\
Viisi villoa keritsi, kuusi lammasta savitsi,             \\
villat saatteli saraksi, kaikki vatvoi vaattehiksi        \\
ennen päivän nousemista, auringon ylenemistä.             \\
                                                          \\
Pesi siitä pitkät pöyät, laajat lattiat lakaisi           \\
vastasella varpaisella, luutasella lehtisellä.            \\
Ammueli rikkasensa vaskisehen vakkasehen;                 \\
vei ne ulos usta myöten, pellolle pihoa myöten,           \\
perimäisen pellon päähän, alimaisen aian suuhun.          \\
Seisattelihe rikoille, kuuntelihe, kääntelihe:            \\
kuulevi mereltä itkun, poikki joen juorotuksen.           \\
                                                          \\
Juosten joutuvi takaisin, pian pirttihin menevi;          \\
sanoi tuonne saatuansa, toimitteli tultuansa:             \\
"Kuulin mie mereltä itkun, poikki joen juorotuksen."      \\
                                                          \\
Louhi, Pohjolan emäntä, Pohjan akka harvahammas,          \\
pian pistihe pihalle, vierähti veräjän suuhun;            \\
siinä korvin kuunteleikse. Sanan virkkoi, noin nimesi:    \\
"Ei ole itku lapsen itku eikä vaimojen valitus;           \\
itku on partasuun urohon, jouhileuan juorottama."         \\
                                                          \\
Työnnälti venon vesille, kolmilaian lainehille;           \\
itse loihe soutamahan. Sekä souti jotta joutui:           \\
souti luoksi Väinämöisen, luoksi itkevän urohon.          \\
                                                          \\
Siinä itki Väinämöinen, urisi Uvannon sulho               \\
pahalla pajupurolla, tiheällä tuomikolla:                 \\
suu liikkui, järisi parta, vaan ei leuka lonkaellut.      \\
                                                          \\
Sanoi Pohjolan emäntä, puhutteli, lausutteli:             \\
"Ohoh sinua, ukko utra! Jo olet maalla vierahalla."       \\
                                                          \\
Vaka vanha Väinämöinen päätänsä kohottelevi.              \\
Sanan virkkoi, noin nimesi: "Jo ma tuon itseki tieän:     \\
olen maalla vierahalla, tuiki tuntemattomalla.            \\
Maallani olin parempi, kotonani korkeampi."               \\
                                                          \\
Louhi, Pohjolan emäntä, sanan virkkoi, noin nimesi:       \\
"Saisiko sanoakseni, oisiko lupa kysyä,                   \\
mi sinä olet miehiäsi ja kuka urohiasi?"                  \\
                                                          \\
Vaka vanha Väinämöinen sanan virkkoi, noin nimesi:        \\
"Mainittihinpa minua, arveltihin aikoinansa               \\
illoilla iloitsijaksi, joka laakson laulajaksi            \\
noilla Väinölän ahoilla, Kalevalan kankahilla.            \\
Mi jo lienenki katala, tuskin tunnen itsekänä."           \\
                                                          \\
Louhi, Pohjolan emäntä, sanan virkkoi, noin nimesi:       \\
"Nouse jo norosta, miesi, uros, uuelle uralle,            \\
haikeasi haastamahan, satuja sanelemahan!"                \\
                                                          \\
Otti miehen itkemästä, urohon urisemasta;                 \\
saattoi siitä purtehensa, istutti venon perähän.          \\
Itse airoille asettui, soutimille suorittihe;             \\
souti poikki Pohjolahan, viepi vierahan tupahan.          \\
                                                          \\
Syötteli nälästynehen, kastunehen kuivaeli;               \\
siitä viikon hierelevi, hierelevi, hautelevi:             \\
teki miehen terveheksi, urohon paranneheksi.              \\
Kysytteli, lausutteli, itse virkki, noin nimesi:          \\
"Mitä itkit, Väinämöinen, uikutit, uvantolainen,          \\
tuolla paikalla pahalla, rannalla meryttä vasten?"        \\
                                                          \\
Vaka vanha Väinämöinen sanan virkkoi, noin nimesi:        \\
"Onpa syytä itkeäni, vaivoja valittoani!                  \\
Kauan oon meriä uinut, lapioinnut lainehia                \\
noilla väljillä vesillä, ulapoilla auke'illa.             \\
                                                          \\
"Tuota itken tuon ikäni, puhki polveni murehin,           \\
kun ma uin omilta mailta, tulin mailta tuttavilta         \\
näille ouoille oville, veräjille vierahille.              \\
Kaikki täällä puut purevi, kaikki havut hakkoavi,         \\
joka koivu koikkoavi, joka leppä leikkoavi:               \\
yks' on tuuli tuttuani, päivä ennen nähtyäni              \\
näillä mailla vierahilla, äkkiouoilla ovilla."            \\
                                                          \\
Louhi, Pohjolan emäntä, siitä tuon sanoiksi saatti:       \\
"Elä itke, Väinämöinen, uikuta, uvantolainen!             \\
Hyvä tääll' on ollaksesi, armas aikaellaksesi,            \\
syöä lohta luotaselta, sivulta sianlihoa."                \\
                                                          \\
Silloin vanha Väinämöinen itse tuon sanoiksi virkki:      \\
"Kylkehen kyläinen syönti hyvissäki vierahissa;           \\
mies on maallansa parempi, kotonansa korkeampi.           \\
Soisipa sula Jumala, antaisipa armoluoja:                 \\
pääsisin omille maille, elomaillen entisille!             \\
Parempi omalla maalla vetonenki virsun alta,              \\
kuin on maalla vierahalla kultamaljasta metonen."         \\
                                                          \\
Louhi, Pohjolan emäntä, sanan virkkoi, noin nimesi:       \\
"Niin mitä minullen annat, kun saatan omille maille,      \\
oman peltosi perille, kotisaunan saapuville?"             \\
                                                          \\
Sanoi vanha Väinämöinen: "Mitäpä kysyt minulta,           \\
jos saatat omille maille, oman peltoni perille,           \\
oman käen kukkumille, oman linnun laulamille!             \\
Otatko kultia kypärin, hope'ita huovallisen?"             \\
                                                          \\
Louhi, Pohjolan emäntä, sanan virkkoi, noin nimesi:       \\
"Ohoh viisas Väinämöinen, tietäjä iän-ikuinen!            \\
En kysele kultiasi, halaja hope'itasi:                    \\
kullat on lasten kukkasia, hopeat hevon helyjä.           \\
Taiatko takoa sammon, kirjokannen kalkutella              \\
joutsenen kynän nenästä, maholehmän maitosesta,           \\
yhen ohrasen jyvästä, yhen uuhen villasesta,              \\
niin annan tytön sinulle, panen neien palkastasi,         \\
saatan sun omille maille, oman linnun laulamille,         \\
oman kukon kuulumille, oman peltosi perille."             \\
                                                          \\
Vaka vanha Väinämöinen sanan virkkoi, noin nimesi:        \\
"Taia en sampoa takoa, kirjokantta kirjoitella.           \\
Saata mie omille maille: työnnän seppo Ilmarisen,         \\
joka samposi takovi, kirjokannet kalkuttavi,              \\
neitosi lepyttelevi, tyttäresi tyy'yttävi.                \\
                                                          \\
"Se on seppo sen mokoma, ylen taitava takoja,             \\
jok' on taivoa takonut, ilman kantta kalkutellut:         \\
ei tunnu vasaran jälki eikä pihtien pitämät."             \\
                                                          \\
Louhi, Pohjolan emäntä, sanan virkkoi, noin nimesi:       \\
"Sille työnnän tyttäreni, sille lapseni lupoan,           \\
joka sampuen takovi, kannen kirjo kirjoittavi             \\
joutsenen kynän nenästä, maholehmän maitosesta,           \\
yhen ohrasen jyvästä, yhen uuhen untuvasta."              \\
                                                          \\
Pani varsan valjahisin, ruskean re'en etehen;             \\
saattoi vanhan Väinämöisen, istutti oron rekehen.         \\
Siitä tuon sanoiksi virkki, itse lausui, noin nimesi:     \\
"Elä päätäsi ylennä, kohottele kokkoasi,                  \\
kun ei uupune oronen, tahi ei ilta ennättäne:             \\
josp' on päätäsi ylennät, kohottelet kokkoasi,            \\
jo toki tuho tulevi, paha päivä päälle saapi."            \\
                                                          \\
Siitä vanha Väinämöinen löi orosen juoksemahan,           \\
harjan liina liikkumahan. Ajoa karittelevi                \\
pimeästä Pohjolasta, summasta Sariolasta.                 \\