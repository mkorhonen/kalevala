% Viides runo

\chapter*{Viides: Runo}

\colorini{J}{o oli sanoma saatu}, viety viesti tuonnemmaksi\\
neien nuoren nukkumasta, kaunihin katoamasta.            \\
                                                         \\
Vaka vanha Väinämöinen, tuo tuosta pahoin pahastui:      \\
itki illat, itki aamut, yöhyet enemmin itki,             \\
kun oli kaunis kaatununna, neitonen nukahtanunna,        \\
mennyt lietohon merehen, alle aaltojen syvien.           \\
                                                         \\
Astui huollen, huokaellen, syämellä synkeällä            \\
rannalle meren sinisen. Sanan virkkoi, noin nimesi:      \\
"Sano nyt, Untamo, unesi, maku'usi, maan venyjä:         \\
missä Ahtola asuvi, neiot Vellamon venyvi?"              \\
                                                         \\
Sanoipa Untamo unensa, maku'unsa maan venyjä:            \\
"Tuolla Ahtola asuvi, neiot Vellamon venyvi:             \\
nenässä utuisen niemen, päässä saaren terhenisen         \\
alla aaltojen syvien, päällä mustien mutien.             \\
                                                         \\
"Siellä Ahtola asuvi, neiot Vellamon venyvi              \\
pikkuisessa pirttisessä, kamarissa kaitaisessa,          \\
kiven kirjavan kylessä, paaen paksun kainalossa."        \\
                                                         \\
Siitä vanha Väinämöinen vetihe venesijoille.             \\
Silmeävi siimojansa, katselevi onkiansa;                 \\
otti ongen taskuhunsa, väkärauan väskyhynsä.             \\
Soutoa melastelevi, päähän saaren saauttavi,             \\
nenähän utuisen niemen, päähän saaren terhenisen.        \\
                                                         \\
Siin' oli ongella olija, aina siimalla asuja,            \\
käeksellä kääntelijä. Laski launihin merelle,            \\
ongitteli, orhitteli: vapa vaskinen vapisi,              \\
hope'inen siima siukui, nuora kultainen kulisi.          \\
                                                         \\
Jo päivänä muutamana, huomenna moniahana                 \\
kala otti onkehensa, taimen takrarautahansa.             \\
Sen veti venosehensa, talui talkapohjahansa.             \\
                                                         \\
Katselevi, kääntelevi. Sanan virkkoi, noin nimesi:       \\
"Onp' on tuo kala kalanen, kun en tuota tunnekana!       \\
Sileähk' on siikaseksi, kuleahka kuujaseksi,             \\
haleahka haukiseksi, evätöin emäkalaksi;                 \\
ihala imehnoksiki, päärivatoin neitoseksi,               \\
vyötöin veen on tyttöseksi, korvitoin kotikanaksi:       \\
luopuisin meriloheksi, syvän aallon ahveneksi."          \\
                                                         \\
Vyöll' on veitsi Väinämöisen, pää hopea huotrasessa.     \\
Veti veitsen viereltänsä, huotrastansa pää hopean        \\
kalan palstoin pannaksensa, lohen leikkaellaksensa       \\
aamuisiksi atrioiksi, murkinaisiksi muruiksi,            \\
lohisiksi lounahiksi, iltaruoiksi isoiksi.               \\
                                                         \\
Alkoi lohta leikkaella, veitsen viilteä kaloa:           \\
lohi loimahti merehen, kala kirjo kimmeltihe             \\
pohjasta punaisen purren, venehestä Väinämöisen.         \\
                                                         \\
Äsken päätänsä ylenti, oikeata olkapäätä                 \\
vihurilla viiennellä, kupahalla kuuennella;              \\
nosti kättä oikeata, näytti jalkoa vasenta               \\
seitsemännellä selällä, yheksännen aallon päällä.        \\
                                                         \\
Sieltä tuon sanoiksi virkki, itse lausui ja pakisi:      \\
"Oi sie vanha Väinämöinen! En ollut minä tuleva          \\
lohi leikkaellaksesi, kala palstoin pannaksesi,          \\
aamuisiksi atrioiksi, murkinaisiksi muruiksi,            \\
lohisiksi lounahiksi, iltaruoiksi isoiksi."              \\
                                                         \\
Sanoi vanha Väinämöinen: "Miksi sie olit tuleva?"        \\
                                                         \\
"Olinpa minä tuleva kainaloiseksi kanaksi,               \\
ikuiseksi istujaksi, polviseksi puolisoksi,              \\
sijasi levittäjäksi, päänalaisen laskijaksi,             \\
pirtin pienen pyyhkijäksi, lattian lakaisijaksi,         \\
tulen tuojaksi tupahan, valkean virittäjäksi,            \\
leivän paksun paistajaksi, mesileivän leipojaksi,        \\
olutkannun kantajaksi, atrian asettajaksi.               \\
                                                         \\
"En ollut merilohia, syvän aallon ahvenia:               \\
olin kapo, neiti nuori, sisar nuoren Joukahaisen,        \\
kuta pyyit kuun ikäsi, puhki polvesi halasit.            \\
                                                         \\
"Ohoh, sinua, ukko utra, vähämieli Väinämöinen,          \\
kun et tuntenut piteä Vellamon vetistä neittä,           \\
Ahon lasta ainokaista!"                                  \\
                                                         \\
Sanoi vanha Väinämöinen alla päin, pahoilla mielin:      \\
"Oi on sisar Joukahaisen! Toki tullos toinen kerta!"     \\
                                                         \\
Eip' on toiste tullutkana, ei toiste sinä ikänä:         \\
jo vetihe, vierähtihe, ve'en kalvosta katosi             \\
kiven kirjavan sisähän, maksankarvaisen malohon.         \\
                                                         \\
Vaka vanha Väinämöinen tuo on tuossa arvelevi,           \\
miten olla, kuin eleä. Jo kutaisi sulkkunuotan,          \\
veti vettä ristin rastin, salmen pitkin, toisen poikki;  \\
veti vienoja vesiä, lohiluotojen lomia,                  \\
noita Väinölän vesiä, Kalevalan kannaksia,               \\
synkkiä syväntehiä, suuria selän napoja,                 \\
Joukolan jokivesiä, Lapin lahtirantasia.                 \\
                                                         \\
Sai kyllin kaloja muita, kaikkia ve'en kaloja,           \\
ei saanut sitä kalaista, mitä mielensä pitävi:           \\
Vellamon vetistä neittä, Ahon lasta ainokaista.          \\
                                                         \\
Siitä vanha Väinämöinen alla päin, pahoilla mielin,      \\
kaiken kallella kypärin itse tuon sanoiksi virkki:       \\
"Ohoh, hullu, hulluuttani, vähämieli, miehuuttani!       \\
Olipa minulla mieltä, ajatusta annettuna,                \\
syäntä suurta survottuna, oli ennen aikoinansa.          \\
Vaanpa nyt tätä nykyä, tällä inhalla iällä,              \\
puuttuvalla polveksella! Kaikki on mieli melkeässä,      \\
ajatukset arvoisessa, kaikki toimi toisialla.            \\
                                                         \\
"Kuta vuotin kuun ikäni, kuta puolen polveani,           \\
Vellamon vetistä neittä, veen on viimeistä tytärtä       \\
ikuiseksi ystäväksi, polviseksi puolisoksi,              \\
                                                         \\
se osasi onkeheni, vierähti venoseheni:                  \\
minä en tuntenut piteä, en kotihin korjaella,            \\
laskin jälle lainehisin, alle aaltojen syvien!"          \\
                                                         \\
Meni matkoa vähäisen, astui huollen, huokaellen;         \\
kulkevi kotia kohti. Sanan virkkoi, noin nimesi:         \\
                                                         \\
"Kukkui muinaiset käkeni, entiset ilokäkeni,             \\
kukkui ennen illoin, aamuin, kerran keskipäivälläki:     \\
mikä nyt sorti suuren äänen, äänen kaunihin kaotti?      \\
Suru sorti suuren äänen, huoli armahan alenti;           \\
sill' ei kuulu kukkuvaksi, päivän laskun laulavaksi      \\
minun iltani iloksi, huomeneni huopeheksi.               \\
                                                         \\
"Enkä nyt tuota tieäkänä miten olla, kuin eleä,          \\
tällä ilmalla asua, näillä mailla matkaella.             \\
Oisiko emo elossa, vanhempani valvehella,                \\
sepä saattaisi sanoa, miten pystössä pysyä,              \\
murehisin murtumatta, huolihin katoamatta                \\
näissä päivissä pahoissa, ape'issa miel'aloissa!"        \\
                                                         \\
Emo hauasta havasi, alta aallon vastaeli:                \\
"Viel' onpi emo elossa, vanhempasi valvehella,           \\
joka saattavi sanoa, miten olla oikeana,                 \\
murehisin murtumatta, huolihin katoamatta                \\
niissä päivissä pahoissa, ape'issa miel'aloissa:         \\
mene Pohjan tyttärihin! Siell' on tyttäret somemmat,     \\
neiet kahta kaunihimmat, viittä, kuutta virkeämmät,      \\
ei Joukon jorottaria, Lapin lapsilönttäreitä.            \\
                                                         \\
"Sieltä naios, poikaseni, paras Pohjan tyttäristä,       \\
jok' on sievä silmiltänsä, kaunis katsannoisiltansa,     \\
aina joutuisa jalalta sekä liukas liikunnolta!"          \\