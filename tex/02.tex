% Toinen runo

\chapter*{Toinen Runo}

\colorini{N}{ousi siitä Väinämöinen} jalan kahen kankahalle				\\
saarehen selällisehen, manterehen puuttomahan.                  \\
                                                                \\
Viipyi siitä vuotta monta, aina eellehen eleli                  \\
saaressa sanattomassa, manteressa puuttomassa.                  \\
                                                                \\
Arvelee, ajattelevi, pitkin päätänsä pitävi:                    \\
kenpä maita kylvämähän, toukoja tihittämähän?                   \\
                                                                \\
Pellervoinen, pellon poika, Sampsa poika pikkarainen,           \\
sep' on maita kylvämähän, toukoja tihittämähän!                 \\
                                                                \\
Kylvi maita kyyhätteli, kylvi maita, kylvi soita,               \\
kylvi auhtoja ahoja, panettavi paasikoita.                      \\
                                                                \\
Mäet kylvi männiköiksi, kummut kylvi kuusikoiksi,               \\
kankahat kanervikoiksi, notkot nuoriksi vesoiksi.               \\
                                                                \\
Noromaille koivut kylvi, lepät maille leyhke'ille,              \\
tuomet kylvi tuorehille, raiat maille raikkahille,              \\
pihlajat pyhille maille, pajut maille paisuville,               \\
katajat karuille maille, tammet virran vieremille.              \\
                                                                \\
Läksi puut ylenemähän, vesat nuoret nousemahan.                 \\
Kasvoi kuuset kukkalatvat, lautui lakkapäät petäjät.            \\
Nousi koivupuut noroilla, lepät mailla leyhke'illä,             \\
tuomet mailla tuorehilla, katajat karuilla mailla,              \\
katajahan kaunis marja, tuomehen hyvä he'elmä.                  \\
                                                                \\
Vaka vanha Väinämöinen kävi tuota katsomahan                    \\
Sampsan siemenen aloa, Pellervoisen kylvämiä.                   \\
Näki puut ylenneheksi, vesat nuoret nousneheksi;                \\
yks' on tammi taimimatta, juurtumatta puu Jumalan.              \\
                                                                \\
Heitti herjan valloillensa, olevillen onnillensa;               \\
vuotti vielä yötä kolme, saman verran päiviäki.                 \\
                                                                \\
Kävi siitä katsomahan viikon päästä viimeistäki:                \\
ei ole tammi kasvanunna, juurtununna puu Jumalan.               \\
                                                                \\
Niin näkevi neljä neittä, viisi veen on morsianta.              \\
Ne oli nurmen niitännässä, kastekorren katkonnassa              \\
nenässä utuisen niemen, päässä saaren terhenisen;               \\
mink' on niitti, sen haravoi, kaikki karhille veteli.           \\
                                                                \\
Tulipa merestä Tursas, uros aalloista yleni.                    \\
Tunki heinäset tulehen, ilmivalkean väkehen;                    \\
ne kaikki poroksi poltti, kypeniksi kyyetteli.                  \\
                                                                \\
Tuli tuhkia läjänen, koko kuivia poroja.                        \\
Saip' on siihen lemmen lehti, lemmen lehti, tammen              \\
terho,                                                          \\
josta kasvoi kaunis taimi, yleni vihanta virpi;                 \\
nousi maasta mansikkaisna, kasvoi kaksihaarukkaisna.            \\
                                                                \\
Ojenteli oksiansa, levitteli lehviänsä.                         \\
Latva täytti taivahalle, lehvät ilmoille levisi:                \\
piätti pilvet juoksemasta, hattarat hasertamasta,               \\
päivän peitti paistamasta, kuuhuen kumottamasta.                \\
                                                                \\
Silloin vanha Väinämöinen arvelee, ajattelevi:                  \\
oisko tammen taittajata, puun sorean sortajata?                 \\
Ikävä inehmon olla, kamala kalojen uia                          \\
ilman päivän paistamatta, kuuhuen kumottamatta.                 \\
                                                                \\
Ei ole sitä urosta eikä miestä urheata,                         \\
joka taisi tammen kaata, satalatvan langettoa.                  \\
                                                                \\
Siitä vanha Väinämöinen itse tuon sanoiksi virkki:              \\
"Kave äiti, kantajani, luonnotar, ylentäjäni!                   \\
Laitapa ve'en väkeä  veessä on väkeä paljo                      \\
tämä tammi taittamahan, puu paha hävittämähän                   \\
eestä päivän paistavaisen, tieltä kuun kumottavaisen!"          \\
                                                                \\
Nousipa merestä miesi, uros aallosta yleni.                     \\
Ei tuo ollut suuren suuri eikä aivan pienen pieni:              \\
miehen peukalon pituinen, vaimon vaaksan korkeuinen.            \\
                                                                \\
Vaski- oli hattu hartioilla, vaskisaappahat jalassa,            \\
vaskikintahat käessä, vaskikirjat kintahissa,                   \\
vaskivyöhyt vyölle vyötty, vaskikirves vyön takana:             \\
varsi peukalon pituinen, terä kynnen korkeuinen.                \\
                                                                \\
Vaka vanha Väinämöinen arvelee, ajattelevi:                     \\
on miesi näkemiänsä, uros silmänluontiansa,                     \\
pystyn peukalon pituinen, härän kynnen korkunainen!             \\
                                                                \\
Siitä tuon sanoiksi virkki, itse lausui, noin nimesi:           \\
"Mi sinä olet miehiäsi, ku, kurja, urohiasi?                    \\
Vähän kuollutta parempi, katonutta kaunihimpi!"                 \\
                                                                \\
Sanoi pikku mies merestä, uros aallon vastaeli:                 \\
"Olen mie mokoma miesi, uros pieni, veen väkeä.                 \\
Tulin tammen taittamahan, puun murskan                          \\
murentamahan."                                                  \\
                                                                \\
Vaka vanha Väinämöinen itse tuon sanoiksi virkki:               \\
"Ei liene sinua luotu, eipä luotu eikä suotu                    \\
ison tammen taittajaksi, puun kamalan kaatajaksi."              \\
                                                                \\
Sai toki sanoneheksi; katsahtavi vielä kerran:                  \\
näki miehen muuttunehen, uuistunehen urohon!                    \\
Jalka maassa teutaroivi, päähyt pilviä pitävi;                  \\
parta on eessä polven päällä, hivus kannoilla takana;           \\
syltä oli silmien välitse, syltä housut lahkehesta,             \\
puoltatoista polven päästä, kahta kaation rajasta.              \\
                                                                \\
Hivelevi kirvestänsä, tahkaisi tasatereä                        \\
kuutehen kovasimehen, seitsemähän sieran päähän.                \\
                                                                \\
Astua lykyttelevi, käyä kulleroittelevi                         \\
lave'illa lahkehilla, leve'illä liehuimilla.                    \\
Astui kerran keikahutti hienoiselle hietikolle,                 \\
astui toisen torkahutti maalle maksankarvaiselle,               \\
kolmannenki koikahutti juurelle tulisen tammen.                 \\
                                                                \\
Iski puuta kirvehellä, tarpaisi tasaterällä.                    \\
Iski kerran, iski toisen, kohta kolmannen yritti;               \\
tuli tuiski kirvehestä, panu tammesta pakeni:                   \\
tahtoi tammi kallistua, lysmyä rutimoraita.                     \\
                                                                \\
Niin kerralla kolmannella jopa taisi tammen kaata,              \\
ruhtoa rutimoraian, satalatvan lasketella.                      \\
Tyven työnnytti itähän, latvan laski luotehesen,                \\
lehvät suurehen suvehen, oksat puolin pohjosehen.               \\
                                                                \\
Kenpä siitä oksan otti, se otti ikuisen onnen;                  \\
kenpä siitä latvan taittoi, se taittoi ikuisen taian;           \\
kenpä lehvän leikkaeli, se leikkoi ikuisen lemmen.              \\
Mi oli lastuja pirannut, pälähellyt pälkäreitä                  \\
selvälle meren selälle, lake'ille lainehille,                   \\
noita tuuli tuuitteli, meren läikkä läikytteli                  \\
venosina veen selällä, laivasina lainehilla.                    \\
                                                                \\
Kantoi tuuli Pohjolahan. Pohjan piika pikkarainen               \\
huntujahan huuhtelevi, virutteli vaattehia                      \\
rannalla vesikivellä pitkän niemyen nenässä.                    \\
                                                                \\
Näki lastun lainehilla; tuon kokosi konttihinsa,                \\
kantoi kontilla kotihin, pitkäkielellä piha'an,                 \\
tehä noian nuoliansa, ampujan asehiansa.                        \\
                                                                \\
Kun oli tammi taittununna, kaatununna puu katala,               \\
pääsi päivät paistamahan, pääsi kuut kumottamahan,              \\
pilvet pitkin juoksemahan, taivon kaaret kaartamahan            \\
nenähän utuisen niemen, päähän saaren terhenisen.               \\
                                                                \\
Siit' alkoi salot silota, metsät mielin kasvaella,              \\
lehtipuuhun, ruohomaahan, linnut puuhun laulamahan,             \\
rastahat iloitsemahan, käki päällä kukkumahan.                  \\
                                                                \\
Kasvoi maahan marjanvarret, kukat kultaiset keolle;             \\
ruohot kasvoi kaikenlaiset, monenmuotoiset sikesi.              \\
Ohra on yksin nousematta, touko kallis kasvamatta.              \\
                                                                \\
Siitä vanha Väinämöinen astuvi, ajattelevi                      \\
rannalla selän sinisen, ve'en vankan vieremillä.                \\
Löyti kuusia jyviä, seitsemiä siemeniä                          \\
rannalta merelliseltä, hienoiselta hietiköltä;                  \\
kätki nää'än nahkasehen, koipehen kesäoravan.                   \\
                                                                \\
Läksi maata kylvämähän, siementä sirottamahan                   \\
vierehen Kalevan kaivon, Osmon pellon penkerehen.               \\
                                                                \\
Tirskuipa tiainen puusta: "Eipä nouse Osmon ohra,               \\
ei kasva Kalevan kaura ilman maan alistamatta,                  \\
ilman kasken kaatamatta, tuon tulella polttamatta."             \\
                                                                \\
Vaka vanha Väinämöinen teetti kirvehen terävän.                 \\
Siitä kaatoi kasken suuren, mahottoman maan alisti.             \\
Kaikki sorti puut soreat; yhen jätti koivahaisen                \\
lintujen leposijaksi, käkösen kukuntapuuksi.                    \\
                                                                \\
Lenti kokko halki taivon, lintunen ylitse ilman.                \\
Tuli tuota katsomahan: "Miksipä on tuo jätetty                  \\
koivahainen kaatamatta, puu sorea sortamatta?"                  \\
                                                                \\
Sanoi vanha Väinämöinen: "Siksipä on tuo jätetty:               \\
lintujen lepeämiksi, kokon ilman istumiksi."                    \\
                                                                \\
Sanoi kokko, ilman lintu: "Hyvinpä sinäki laait:                \\
heitit koivun kasvamahan, puun sorean seisomahan                \\
linnuille lepeämiksi, itselleni istumiksi."                     \\
                                                                \\
Tulta iski ilman lintu, valahutti valkeaista.                   \\
Pohjaistuuli kasken poltti, koillinen kovin porotti:            \\
poltti kaikki puut poroksi, kypeniksi kyyetteli.                \\
                                                                \\
Siitä vanha Väinämöinen otti kuusia jyviä,                      \\
seitsemiä siemeniä yhen nää'än nahkasesta,                      \\
koivesta kesäoravan, kesäkärpän kämmenestä.                     \\
                                                                \\
Läksi maata kylvämähän, siementä sirottamahan.                  \\
Itse tuon sanoiksi virkki: "Minä kylvän kyyhättelen             \\
Luojan sormien lomitse, käen kautta kaikkivallan                \\
tälle maalle kasvavalle, ahollen ylenevälle.                    \\
                                                                \\
"Akka manteren-alainen, mannun eukko, maan emäntä!              \\
Pane nyt turve tunkemahan, maa väkevä vääntämähän!              \\
Eip' on maa väkeä puutu sinä ilmoisna ikänä,                    \\
kun lie armo antajista, lupa luonnon tyttäristä.                \\
                                                                \\
"Nouse, maa, makoamasta, Luojan nurmi, nukkumasta!              \\
Pane korret korttumahan sekä varret varttumahan!                \\
Tuhansin neniä nosta, saoin haaroja hajota                      \\
kynnöstäni, kylvöstäni, varsin vaivani näöstä!                  \\
                                                                \\
"Oi Ukko, ylijumala tahi taatto taivahinen,                     \\
vallan pilvissä pitäjä, hattarojen hallitsija!                  \\
Piä pilvissä keräjät, sekehissä neuvot selvät!                  \\
Iätä iästä pilvi, nosta lonka luotehesta,                       \\
toiset lännestä lähetä, etelästä ennättele!                     \\
Vihmo vettä taivosesta, mettä pilvistä pirota                   \\
orahille nouseville, touoille tohiseville!"                     \\
                                                                \\
Tuo Ukko, ylijumala, taatto taivon valtiainen,                  \\
piti pilvissä keräjät, sekehissä neuvot selvät.                 \\
Iätti iästä pilven, nosti longan luotehesta,                    \\
toisen lännestä lähetti, etelästä ennätteli;                    \\
syrjin yhtehen sysäsi, lomituksin loukahutti.                   \\
Vihmoi vettä taivosesta, mettä pilvistä pirotti                 \\
orahille kasvaville, touoille tohiseville.                      \\
Nousipa oras okinen, kannonkarvainen yleni                      \\
maasta pellon pehmeästä, Väinämöisen raatamasta.                \\
                                                                \\
Jopa tuosta toisna päänä, kahen, kolmen yön perästä,            \\
viikon päästä viimeistäki vaka vanha Väinämöinen                \\
kävi tuota katsomahan kyntöänsä, kylvöänsä,                     \\
varsin vaivansa näköä: kasvoi ohra mieltä myöten,               \\
tähkät kuuella taholla, korret kolmisolmuisena.                 \\
                                                                \\
Siinä vanha Väinämöinen katseleikse, käänteleikse.              \\
Niin tuli kevätkäkönen, näki koivun kasvavaksi:                 \\
"Miksipä on tuo jätetty koivahainen kaatamatta?"                \\
                                                                \\
Sanoi vanha Väinämöinen: "Siksipä on tuo jätetty                \\
koivahainen kasvamahan: sinulle kukuntapuuksi.                  \\
Siinä kukkuos, käkönen, helkyttele, hietarinta,                 \\
hoiloa, hopearinta, tinarinta, riukuttele!                      \\
Kuku illoin, kuku aamuin, kerran keskipäivälläki,               \\
ihanoiksi ilmojani, mieluisiksi metsiäni,                       \\
rahaisiksi rantojani, viljaisiksi vieriäni!"                    \\