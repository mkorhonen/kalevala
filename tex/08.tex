% Kahdeksas Runo

\chapter*{Kahdeksas: Runo}

\colorini{T}{uo oli kaunis Pohjan neiti}, maan kuulu, ve'en valio.\\
Istui ilman vempelellä, taivon kaarella kajotti                 \\
pukehissa puhta'issa, valke'issa vaattehissa;                   \\
kultakangasta kutovi, hope'ista huolittavi                      \\
kultaisesta sukkulasta, pirralla hope'isella.                   \\
                                                                \\
Suihki sukkula piossa, käämi käessä kääperöitsi,                \\
niiet vaskiset vatisi, hope'inen pirta piukki                   \\
neien kangasta kutoissa, hope'ista huolittaissa.                \\
                                                                \\
Vaka vanha Väinämöinen ajoa karittelevi                         \\
pimeästä Pohjolasta, summasta Sariolasta.                       \\
Ajoi matkoa palasen, pikkaraisen piirrätteli:                   \\
kuuli sukkulan surinan ylähältä päänsä päältä.                  \\
                                                                \\
Tuossa päätänsä kohotti, katsahtavi taivahalle:                 \\
kaari on kaunis taivahalla, neiti kaaren kannikalla,            \\
kultakangasta kutovi, hope'ista helkyttävi.                     \\
                                                                \\
Vaka vanha Väinämöinen heti seisatti hevosen.                   \\
Tuossa tuon sanoiksi virkki, itse lausui, noin nimesi:          \\
"Tule, neiti, korjahani, laskeite rekoseheni!"                  \\
                                                                \\
Neiti tuon sanoiksi virkki, itse lausui ja kysyvi:              \\
"Miksi neittä korjahasi, tyttöä rekosehesi?"                    \\
                                                                \\
Vaka vanha Väinämöinen tuop' on tuohon vastaeli:                \\
"Siksi neittä korjahani, tyttöä rekoseheni:                     \\
mesileivän leipojaksi, oluen osoajaksi,                         \\
joka lautsan laulajaksi, ikkunan iloitsijaksi                   \\
noilla Väinölän tiloilla, Kalevalan kartanoilla."               \\
                                                                \\
Neiti tuon sanoiksi virkki, itse lausui ja pakisi:              \\
"Kun kävin mataramaalla, keikuin keltakankahalla                \\
eilen iltamyöhäsellä, aletessa aurinkoisen,                     \\
lintu lauleli lehossa, kyntörastas raksutteli:                  \\
lauleli tytärten mielen ja lauloi miniän mielen.                \\
                                                                \\
"Mie tuota sanelemahan, linnulta kyselemähän:                   \\
'Oi sie kyntörastahainen! Laula korvin kuullakseni:             \\
kumman on parempi olla, kumman olla kuuluisampi,                \\
tyttärenkö taattolassa vai miniän miehelässä?'                  \\
                                                                \\
"Tiainenpa tieon antoi, kyntörastas raksahutti:                 \\
'Valkea kesäinen päivä, neitivalta valkeampi;                   \\
vilu on rauta pakkasessa, vilumpi miniävalta.                   \\
Niin on neiti taattolassa, kuin marja hyvällä maalla,           \\
niin miniä miehelässä, kuin on koira kahlehissa.                \\
Harvoin saapi orja lemmen, ei miniä milloinkana.'"              \\
                                                                \\
Vaka vanha Väinämöinen itse tuon sanoiksi virkki:               \\
"Tyhjiä tiaisen virret, rastahaisen raksutukset!                \\
Lapsi on tytär kotona, vasta on neiti naituansa.                \\
Tule, neiti, korjahani, laskeite rekoseheni!                    \\
En ole mitätön miesi, uros muita untelompi."                    \\
                                                                \\
Neiti taiten vastaeli, sanan virkkoi, noin nimesi:              \\
"Sitte sun mieheksi sanoisin, urohoksi arveleisin,              \\
jospa jouhen halkaiseisit veitsellä kärettömällä,               \\
munan solmuhun vetäisit solmun tuntumattomaksi."                \\
                                                                \\
Vaka vanha Väinämöinen jouhen halki halkaisevi                  \\
veitsellä kärettömällä, aivan tutkaimettomalla;                 \\
munan solmuhun vetävi solmun tuntumattomaksi.                   \\
Käski neittä korjahansa, tyttöä rekosehensa.                    \\
                                                                \\
Neiti taiten vastaeli: "Ehkäpä tulen sinulle,                   \\
kun kiskot kivestä tuohta, säret jäästä aiaksia                 \\
ilman palan pakkumatta, pilkkehen pirahtamatta."                \\
                                                                \\
Vaka vanha Väinämöinen ei tuosta kovin hätäile:                 \\
kiskoipa kivestä tuohta, särki jäästä aiaksia                   \\
ilman palan pakkumatta, pilkkehen pirahtamatta.                 \\
Kutsui neittä korjahansa, tyttöä rekosehensa.                   \\
                                                                \\
Neiti taiten vastoavi, sanovi sanalla tuolla:                   \\
"Sillenpä minä menisin, kenp' on veistäisi venosen              \\
kehrävarteni muruista, kalpimeni kappaleista,                   \\
työntäisi venon vesille, uuen laivan lainehille                 \\
ilman polven polkematta, ilman kouran koskematta,               \\
käsivarren kääntämättä, olkapään ojentamatta."                  \\
                                                                \\
Siitä vanha Väinämöinen itse tuon sanoiksi virkki:              \\
"Liene ei maassa, maailmassa, koko ilman kannen alla            \\
mointa laivan laatijata, vertoani veistäjätä."                  \\
                                                                \\
Otti värttinän muruja, kehrävarren kiertimiä;                   \\
läksi veistohon venosen, satalauan laittelohon                  \\
vuorelle teräksiselle, rautaiselle kalliolle.                   \\
                                                                \\
Veikaten venettä veisti, purtta puista uhkaellen.               \\
Veisti päivän, veisti toisen, veisti kohta kolmannenki:         \\
ei kirves kivehen koske, kasa ei kalka kalliohon.               \\
                                                                \\
Niin päivällä kolmannella Hiisi pontta pyörähytti,              \\
Lempo tempasi tereä, Paha vartta vaapahutti.                    \\
Kävipä kivehen kirves, kasa kalkkoi kalliohon;                  \\
kirves kilpistyi kivestä, terä liuskahti liha'an,               \\
polvehen pojan pätöisen, varpahasen Väinämöisen.                \\
Sen Lempo lihoille liitti, Hiisi suonille sovitti:              \\
veri pääsi vuotamahan, hurme huppelehtamahan.                   \\
                                                                \\
Vaka vanha Väinämöinen, tietäjä iän-ikuinen,                    \\
tuossa tuon sanoiksi virkki, noin on lausui ja pakisi:          \\
"Oi sie kirves kikkanokka, tasaterä tapparainen!                \\
Luulitko puuta purrehesi, honkoa hotaisnehesi,                  \\
petäjätä pannehesi, koivua kohannehesi,                         \\
kun sa lipsahit liha'an, solahutit suonilleni?"                 \\
                                                                \\
Loihe siitä loitsimahan, sai itse sanelemahan.                  \\
Luki synnyt syitä myöten, luottehet lomia myöten,               \\
mutt' ei muista muutamia rauan suuria sanoja,                   \\
joista salpa saataisihin, luja lukko tuotaisihin                \\
noille rauan ratkomille, suu sinervän silpomille.               \\
                                                                \\
Jo veri jokena juoksi, hurme koskena kohisi:                    \\
peitti maassa marjan varret, kanervaiset kankahalla.            \\
Eik' ollut sitä mätästä, jok' ei tullut tulvillehen             \\
noita liikoja veriä, hurmehia huurovia                          \\
polvesta pojan totisen, varpahasta Väinämöisen.                 \\
                                                                \\
Vaka vanha Väinämöinen ketti villoja kiveltä,                   \\
otti suolta sammalia, maasta mättähän repäisi                   \\
tukkeheksi tuiman reiän, paikaksi pahan veräjän;                \\
ei vääjä vähäistäkänä, pikkuistakana piätä.                     \\
                                                                \\
Jopa tuskaksi tulevi, läylemmäksi lankeavi.                     \\
Vaka vanha Väinämöinen itse itkuhun hyräytyi;                   \\
pani varsan valjahisin, ruskean re'en etehen,                   \\
siitä reuoikse rekehen, kohennaikse korjahansa.                 \\
                                                                \\
Laski virkkua vitsalla, helähytti helmisvyöllä;                 \\
virkku juoksi, matka joutui, reki vieri, tie lyheni.            \\
Jo kohta kylä tulevi: kolme tietä kohtoavi.                     \\
                                                                \\
Vaka vanha Väinämöinen ajavi alinta tietä                       \\
alimaisehen talohon. Yli kynnyksen kysyvi:                      \\
"Oisiko talossa tässä rauan raannan katsojata,                  \\
uron tuskan tuntijata, vammojen vakittajata?"                   \\
                                                                \\
Olipa lapsi lattialla, poika pieni pankon päässä.               \\
Tuop' on tuohon vastoavi: "Ei ole talossa tässä                 \\
rauan raannan katsojata, uron tuskan tuntijata,                 \\
kivun kiinniottajata, vammojen vakittajata;                     \\
onpi toisessa talossa: aja toisehen talohon!"                   \\
                                                                \\
Vaka vanha Väinämöinen laski virkkua vitsalla,                  \\
ajoa suhuttelevi. Ajoi matkoa palasen,                          \\
keskimäistä tietä myöten keskimäisehen talohon.                 \\
Kysyi kynnyksen takoa, anoi alta ikkunaisen:                    \\
"Oisiko talossa tässä rauan raannan katsojata,                  \\
salpoa verisatehen, suonikosken sortajata?"                     \\
                                                                \\
Akka oli vanha vaipan alla, kielipalku pankon päässä.           \\
Akka varsin vastaeli, hammas kolmi kolkkaeli:                   \\
"Ei ole talossa tässä rauan raannan katsojata,                  \\
verisynnyn tietäjätä, kivun kiinniottajata;                     \\
onpi toisessa talossa: aja toisehen talohon!"                   \\
                                                                \\
Vaka vanha Väinämöinen laski virkkua vitsalla,                  \\
ajoa suhuttelevi. Ajoi matkoa palasen,                          \\
ylimäistä tietä myöten ylimäisehen talohon.                     \\
Yli kynnyksen kysyvi, lausui lakkapuun takoa:                   \\
"Oisiko talossa tässä rauan raannan katsojata,                  \\
tämän tulvan tukkijata, veren summan sulkijata?"                \\
                                                                \\
Ukko oli uunilla asuva, halliparta harjun alla.                 \\
Ukko uunilta urahti, halliparta paukutteli:                     \\
"On sulettu suuremmatki, jalommatki jaksettuna                  \\
Luojan kolmella sanalla, syvän synnyn säätämällä;               \\
joet suista, järvet päistä, virrat niskalta vihaiset,           \\
lahet niemien nenistä, kannakset kape'immilta."                 \\